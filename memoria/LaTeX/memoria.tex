%%%%%%%%%%%%%%%%%%%%%%%%%%%%%%%%%%%%%%%%%%%%%%%%%%%
%% LaTeX book template                           %%
%% Author:  Amber Jain (http://amberj.devio.us/) %%
%% License: ISC license                          %%
%%%%%%%%%%%%%%%%%%%%%%%%%%%%%%%%%%%%%%%%%%%%%%%%%%%

\documentclass[a4paper,11pt]{book}
\usepackage[a4paper,margin=0.7in,footskip=0.3in]{geometry}
\usepackage[T1]{fontenc}
\usepackage[utf8]{inputenc}
\usepackage{lmodern}
\usepackage{enumerate}
\usepackage{float}
\setlength{\parindent}{0cm}
\usepackage{titling}
\usepackage{eurosym}
\usepackage{pgfgantt}
\usepackage{lscape}
\usepackage{lmodern,textcomp}

\usepackage{subcaption}
\usepackage{import} 
\usepackage{hyperref}
\usepackage{graphicx}
\usepackage[spanish]{babel}
\usepackage{graphicx}
\graphicspath{ {imagenes/} }
\usepackage{wrapfig}
\usepackage{array}
\newcolumntype{L}{>{\centering\arraybackslash}m{3cm}}
\usepackage{listings}
\usepackage{xcolor}
\definecolor{commentgreen}{RGB}{2,112,10}
\definecolor{stringgreen}{RGB}{2,150,100}

\lstset { %
    language=C++,
    backgroundcolor=\color{black!5}, % set backgroundcolor
    basicstyle=\footnotesize,% basic font setting
    frame=tb, % draw a frame at the top and bottom of the code block
    tabsize=4, % tab space width
    showstringspaces=false, % don't mark spaces in strings
    numbers=left, % display line numbers on the left
    commentstyle=\color{commentgreen}, % comment color
    keywordstyle=\color{blue}, % keyword color
    stringstyle=\color{stringgreen} % string color
}

\input{funciones}
\iffalse
%%%%%%%%%%%%%%%%%%%%%%%%%%%%%%%%%%%%%%%%%%%%%%%%%%%
%PORTADA                         
%%%%%%%%%%%%%%%%%%%%%%%%%%%%%%%%%%%%%%%%%%%%%%%%%%%
%nota, para poner nota al pie de pagina se pone despues de } lo siguiente "\footnote{This is a footnote.}"
% Book's title and subtitle
\title{\Huge \textbf{Estudio del proceso de obtención de puntos clave en nubes de puntos}   \\ \huge Reproducción de algoritmos de PCL y aceleración por hardware digital}
% Author
\author{\textsc{Javier Pina De Paz}} %
\fi


%%%%%%%otro titulo%%%%%%%%%%%
\pretitle{%
  \begin{center}
  \LARGE
  \includegraphics[]{politecnica}\\[\bigskipamount]
}
\posttitle{\end{center}}




\begin{document}


\title{\Huge \textbf{Estudio del proceso de obtención de puntos clave en nubes de puntos}   \\ \huge Reproducción de algoritmos de PCL y aceleración por hardware digital}
\author{\textsc{Javier Pina De Paz}}
\date{02/02/2019} 


\frontmatter
\maketitle

\iffalse
\begin{figure}[!htb]
\centering
\includegraphics[scale=0.8]{politecnica}
\end{figure}
\fi

%%%%%%%%%%%%%%%%%%%%%%%%%%%%%%%%%%%%%%%%%%%%%%%%%%%%%%%%%%%%%%%
%DEDICATORIA
%%%%%%%%%%%%%%%%%%%%%%%%%%%%%%%%%%%%%%%%%%%%%%%%%%%%%%%%%%%%%%%
\begin{dedication}
Agradezco el apoyo y cariño que he encontrado siempre en mi familia, mi madre y mi hermana que nunca me han fallado y en quienes siempre puedo encontrar mi hogar. 


Me siento afortunado de haber conocido a personas con talento, inteligentes y con buen corazón siendo algunas de ellas ahora mis amigos, gracias por todos los momentos que hemos pasado juntos y los que están por venir.


Agradezco la labor, el apoyo y ayuda de mi tutor de quien también he tomado clases y he podido aprender mucho así como estoy agradecido a todos los docentes involucrados en mi formación durante el grado y máster, sin ellos ni todas las personas que forman parte de mi vida, no podría estar donde estoy ni ser quien soy.

\end{dedication}



\begin{center}
\Huge{Resumen}
\end{center}

A partir de la invención del láser y el desarrollo de sistemas con mayor capacidad de computación y almacenamiento de información, han proliferado   herramientas y aplicaciones basadas en la visión por computador lo que involucra un proceso de adquisición de información, procesamiento de la misma y efectuar una acción en consecuencia si fuera necesario. Este tipo de herramientas aportan nuevas funcionalidades y capacidades tanto en la industria como en la investigación pues ofrecen métodos potentes para analizar y trasladar de la realidad a un ordenador tanto objetos sencillos como entornos exteriores amplios con gran cantidad de detalles.
\\
\\
Una herramienta muy extendida para operaciones de visión por computador es la nube de puntos: conjunto de puntos distribuidos en un espacio tridimensional de manera que crean una superficie discontinua, es decir, puntual, y que son capaces de recrear entornos y objetos. 
\\
\\
Existe una gran variedad de sensores capaces de crear nubes de puntos a partir de su entorno y cuyas diferencias radican en la cantidad de puntos que son capaces de generar y el tiempo que necesitan para ello dando como resultado nubes de puntos de diferentes resoluciones y niveles de detalle tal y como se puede apreciar en las figuras \ref{fig:bunny_simple} y \ref{fig:bunny}

\begin{figure}[!htb]
\minipage{0.48\textwidth}
  \includegraphics[scale=1.0]{bunny_simple_copia}
  \caption{Nube de puntos que representa un conejo.
  Peso total de la nube: 10.6KB.
  Número total de puntos: 397.}\label{fig:bunny_simple}
\endminipage\hfill
\minipage{0.48\textwidth}
  \includegraphics[scale=0.35]{bunny}
  \caption{Nube de puntos que representa un conejo.
  Peso total de la nube: 42.6KB.
  Número total de puntos: 3400.}\label{fig:bunny}
\endminipage\hfill
\end{figure}

Una vez se dispone de una o varias nubes de puntos almacenadas en un computador, es usual realizar algún tipo de operaciones con ellas puesto que generalmente requieren algún tipo de procesamiento para que sean verdaderamente útiles.
\\
\\
Existe una librería de uso gratuito llamada PCL (Point Cloud Library del inglés librería de nubes de puntos) para el procesamiento de nubes de puntos. PCL dispone de algoritmos involucrados en el proceso de alineamiento de nubes de puntos: póngase que se desea crear una nube de puntos a partir de la fachada de un edificio de 200 metros de largo y el sensor debe situarse a una distancia del edificio tal que es capaz de generar nubes que abarcan hasta 120 metros de la fachada. Para crear una única nube de puntos que contenga la totalidad de la fachada del edificio se genera una primera nube que contiene un primer tramo de la fachada, se mueve el sensor lateralmente, se genera otra nube de puntos y se fusionan las dos nubes. 
\\
Para poder fusionar dos o más nubes hay que considerar que éstas tienen puntos semejantes o solapamientos, es decir, hay puntos en ellas que son idénticos. Estos puntos son los llamados puntos clave o keypoints y sirven como enlace para que dos nubes tomadas desde diferentes posiciones, como se ha planteado en el ejemplo, formen una sola.
\\
El ejemplo mencionado se puede apreciar en la figura \ref{fig:fachada} donde la zona de color naranja es el solapamiento de las dos nubes de puntos que conforman la totalidad de la fachada.
\\
\\
\begin{figure}
\centering
\includegraphics[scale=1.0]{fachada}
  \caption{Representación de la creación de dos nubes de puntos sobre una fachada de 200 metros.}\label{fig:fachada}
\end{figure}
En este trabajo se va a estudiar la parte de la librería PCL involucrada con la estimación de keypoints y que es necesaria para el alineamiento de nubes de puntos. Con ello se pretende determinar cuál del los algoritmos ejecutados en software que intervienen en este proceso  necesita más tiempo para ejecutarse. A continuación, se procederá a acelerar dicho algoritmo mediante hardware y se demostrará su correcto funcionamiento sobre un sistema embebido.
\\
\\
Palabras clave: Nube, puntos, clave, keypoint, hardware, software, aceleración, estimación, embebido.



%%%%%%%%%%%%%%%%%%%%%%%%%%%%%%%%%%%%%%%%%%%%%%%%%%%%%%%%%%%%%%%%%%%%%%%%INDICE AUTOGENERADO
%%%%%%%%%%%%%%%%%%%%%%%%%%%%%%%%%%%%%%%%%%%%%%%%%%%%%%%%%%%%%%%%%%%%%%%%
\tableofcontents
%\listoftables    "Esto lo que hace es otra pagina despues del índice con las tablas

\mainmatter  %esto no sé qué hace pero lo dejo por si acaso



\chapter{ Capítulo introductorio}

\begin{chapquote}{Javier, \textit{Todos los días}}
`` Me cago en Dios''
\end{chapquote}

\section{La visión como herramienta}

\subsection{La visión en el ser humano}
El ser humano es capaz de interactuar con el mundo que le rodea de diversas maneras ya sean creativas, constructivas o banales. Esto se puede conseguir mediante los sentidos; Tacto, gusto, oído, olfato y por último pero no menos importante la visión. Este último sentido nos permite recoger gran cantidad de información de nuestro entorno, procesarla y actuar en consecuencia. Es entonces el ojo humano una herramienta de gran valor y elevada precisión que poco tiene que envidiar al órgano análogo en el resto de especies en todo el reino animal. 

Lejos de realizar un estudio anatómico del sentido de la vista en el ser humano, se disponen a continuación ciertas características de interés que permiten apreciar toda su potencia.

Una de las primeras cuestiones que se vienen a la mente cuando se habla de la "simple vista" es su resolución, es decir, cuál es la mínima distancia que es capaz de distinguir. Se puede estudiar mediante la resolución angular que para el ser humano se encuentra entorno a 1' o 2' o l que es lo mismo un intervalo de 0,02º a 0,03º\cite{simple_vista}. Dicho de otra forma, un ojo humano sano y correctamente desarrollado puede distinguir objetos de entre 30cm y 60cm a 1km de distancia. Por ejemplo, una persona podría detectar dos balones de playa de un diámetro aproximado de 45cm hasta 1km de distancia.

%https://es.wikipedia.org/wiki/Simple_vista

El campo de visión es también un aspecto de gran importancia ya que determina la cantidad de información que los ojos pueden recibir en un instante dado. El "cono de visión" queda entorno a unos 130º en vertical y 160º en horizontal que equivale aproximadamente a una lente con longitud focal de 2mm\cite{angle_of_view}. Se entiende como distancia focal, en el ámbito de la fotografía, la distancia entre el punto en el que convergen los rayos de luz formando una imagen y el sensor digital del dispositivo. Esta medida se determina cuando la lente está enfocada al infinito.

En resumidas cuentas, a mayor distancia focal, más estrecho será el ángulo de visión y mayor la magnificación ocurriendo el efecto contrario según se reduce la distancia focal. Se muestra en la figura \ref{fig:focal_length} un esquema de lo explicado: 


\begin{figure}[!htb]
\centering
\minipage{0.7\textwidth}
  \includegraphics[width=\linewidth]{focal_length}
  \caption{Ángulo de visión en función de la distancia focal}\label{fig:focal_length}
\endminipage\hfill

\end{figure}

%https://en.wikipedia.org/wiki/Angle_of_view#Common_lens_angles_of_view


Por último, se considera la velocidad de enfoque o acomodación del ojo humano. Según un experimento llevado cabo por el MIT\cite{mit_experiment} (Massachussets Institute of Technology) el ojo humano es capaz de captar una imagen para su posterior procesamiento en tan solamente 13 milisegundos.

Este experimento consistía en  mostrar un conjunto de seis o doce imágenes y de entre las cuales se pedía a los participantes identificar "una pareja sonriente" o un entorno de campo. 
Para llegar al resultado actual, el equipo del MIT comenzó proyectando imágenes a una velocidad de exposición de 80 milisegundos. Posteriormente se redujo poco a poco este tiempo solamente para las imágenes que había que identificar pasando a 53 milisegundos, luego de 40 a 27, y finalmente se llegaron a los 13 milisegundos. Posteriores reducciones del tiempo de exposición hacían imposible identificar las imágenes.

%http://www.sophimania.pe/ciencia/cerebro-y-neurociencias/estudio-revela-que-el-ojo-humano-capta-una-imagen-en-13-milisegundos/
%http://news.mit.edu/2014/in-the-blink-of-an-eye-0116

Se ha planteado entonces una situación en la que hay un pequeño problema, un instrumento de gran precisión y prolongada durabilidad embarcado en un ser vivo falible y errático. No sólo esto sino que nunca se llega a dominar por completo el manejo de éste órgano así como su uso en conjunto a las demás partes del cuerpo y siempre todo ello limitado por la velocidad de respuesta que puede proporcionar el sistema nervioso.
Bienes sabido que los deportistas de alta competición deben poseer una capacidad de reacción casi inmediata a diferentes tipos de estímulos como lo son sonoros o visuales, es decir, una bocina o un semáforo, por ejemplo. 

Existen dos tipos de tiempos de reacción\cite{tiempo_reaccion}; simple y complejo. El primero hace referencia al tiempo que transcurre desde la recepción de un estímulo anticipado o ya conocido hasta la reacción, mientras que el segundo se diferencia en el tipo de estímulo ya que en este caso es desconocido o se considera un imprevisto. Resulta obvio que el tiempo de reacción simple es menor que el complejo porque la respuesta está decidida de antemano. 

Además, se ha de considerar que la respuesta a estímulos sonoros es más rápida que a los estímulos visuales. Esto se debe a que los receptores auditivos se estimulan mecánicamente (vibración de los huesos del oído debido a una onda sonora) mientras que los visuales lo hacen de forma química.

Habiendo visto estas consideraciones y tomando como referencia los datos de Mateyev 1977 y los estudios de Cometti se disponen en la tabla  datos relevantes sobre tiempos de reacción en atletas y no atletas.

ALTO NIVEL      SONORA  0,05 – 010
LUMINOSA  0,10 – 0,20

NO ATLETAS      SONORA  0,15 – 0,25 y más
LUMINOSA  0,20 – 0,35 Y MÁS

Es decir, en el mejor de los casos y para un atleta profesional, se tarda un quinto de segundo en reaccionar a un estímulo predefinido y por tanto, sabiendo la reacción que hay que llevar a cabo tal y como puede ser poner en movimiento un vehículo al ver la luz verde en un semáforo.
%https://www.fuerzaycontrol.com/la-velocidad-de-reaccion-el-tiempo-de-reaccion-simple-complejo-la-anticipacion/

Aunque la vista en conjunto con el sistema nervioso nos permiten llevar a cabo, en ocasiones, tareas impensables como se pueden ver en deportes de alta competición o en la prestidigitación, por ejemplo, podemos manejar una herramienta todavía más potente, la inteligencia. Entre las muchas definiciones del mencionado concepto se puede ver la siguiente:

"Capacidad de tomar decisiones para resolver problemas"


Esta idea está inevitablemente ligada al término "tecnología" que se puede introducir como:

"La aplicación de conocimientos científicos para la resolución de problemas mediante el diseño y creación de bienes y servicios" 

junto a su etimología, pues se trata de una palabra de origen griego 
%τεχνολογία 
formada por 
%τέχνη 
(arte, técnica u oficio) y 
%λογία 
(el estudio de algo) es fácil apreciar que la evolución y desarrollo del ser humano junto a su inteligencia durante toda su historia han venido ligadas a un progreso tecnológico\cite{historia_tecnologia} incesante, vivo y capaz de abrirse paso incluso en las épocas más oscuras de la historia de la humanidad como se ve en la figura \ref{fig:history_technology} 

\begin{figure}[!htb]
\centering
\minipage{0.7\textwidth}
  \includegraphics[width=\linewidth]{history_technology}
  \caption{Línea temporal de los pregresos tecnológicos más importantes de la hiistoria}\label{fig:history_technology}
\endminipage\hfill
\end{figure}
%https://tecnomagazine.net/2018/04/30/historia-de-la-tecnologia/

Es concretamente en el siglo XX cuando aparecen inventos revolucionarios, que cambian el curso de la historia marcando un definido punto de inflexión en la misma. Véanse por ejemplo la radio, el transistor, o internet. Uno de los impactos más notables es el incremento de la población mundial debido al progreso tecnológico entre otros factores. Se unifican ambos conceptos en la figura \ref{fig:technology_population}

\begin{figure}[!htb]
\centering
\minipage{0.65\textwidth}
  \includegraphics[width=\linewidth]{technology_population}
  \caption{Evolución conjunta de la población mundial y tecnología}\label{fig:technology_population}
\endminipage\hfill
\end{figure}

\subsection{Exportando la capacidad de visión}

Una vez descritas brevemente la potencia, capacidad y posibilidades que aporta la visión en el ser humano junto a su capacidad de resolver problemas mediante la tecnología, se puede ir más allá, subir un nivel y dotar a entes ajenos al hombre de la habilidad de obtener y procesar información visual. Este concepto existe desde finales de la década de los sesenta y puede introducirse como visión artificial o visión por computador\cite{vision_artificial}:
\\
“Disciplina científica que incluye métodos para adquirir, procesar, analizar y comprender las imágenes del mundo real con el fin de producir información numérica o simbólica para que puedan ser tratados por un computador”
\\
Esta forma de trabajar con información visual es posible debido a la puesta en conjunto de diferentes campos como la geometría, física o estadística y demás herramientas que se explicarán más adelante.
%https://es.wikipedia.org/wiki/Visi%C3%B3n_artificial#Detecci%C3%B3n_de_objetos

Bien es cierto que se plantea un problema ya que la forma en que el ojo humano percibe el mundo no es la misma en la que lo hace una máquina pues se pueden establecer sus diferencias como las mismas que hay entre una señal analógica y otra digital, respectivamente.
\\
Véase en primer lugar la mencionada diferencia respecto a las señales:

\begin{itemize}
\item Señal analógica
\\
Se trata de la representación de una magnitud física tal y como se percibe del entorno o se genera con algún instrumento y puede verse como una función matemática continua.
La mayoría de las señales que se perciben son analógicas como ejemplo la intensidad de corriente eléctrica, la temperatura, el sonido, presión y energía.
De este modo existen señales analógicas periódicas (caracterizadas por amplitud y frecuencia) no periódicas (toman cualquier valor independientemente del tiempo). Un ejemplo de señal analógica periódica\cite{señal_analogica} se ve en la figura \ref{fig:analog_signal}

\begin{figure}[!htb]
\centering
\minipage{0.65\textwidth}
  \includegraphics[width=\linewidth]{analog_signal}
  \caption{Señal analógica que representa una onda sonora bajo el mar. Se muestran tres formas de caracterizar su intensidad: Valor de cero a pico, valor de pico a pico y raíz cuadrada del valor medio.}\label{fig:analog_signal}
  
\endminipage\hfill
\end{figure}
%https://dosits.org/science/advanced-topics/introduction-to-signal-levels/

\item Señal digital
\\
Existe una extendida confusión en lo referente a la diferencia entre señal digital y señal discretizada. Se puede partir de una señal analógica senoidal y entonces llegar a una señal discretizada tomando valores equidistantes en el tiempo como se muestra en la figura \ref{fig:analogica_digitalizada}:
\begin{figure}[!htb]
\centering
\minipage{0.65\textwidth}
  \includegraphics[width=\linewidth]{analogica_digitalizada}
  \caption{Señal analógica discretizada. En cada instante de tiempo se tiene un valor perteneciente a un conjunto fijo.}\label{fig:analogica_digitalizada}
\endminipage\hfill
\end{figure}
%https://learn.sparkfun.com/tutorials/analog-vs-digital/digital-signals

Para obtener una señal digital, se codifican cada uno de los valores discretos de la señal representada anteriormente. De este modo se tiene un número finito de valores pertenecientes a un conjunto y no intervalos.

Se puede mostrar en la figura \ref{fig:adc} el funcionamiento básico de un convertidor Analógico/digital para comprender la diferencia entre estos dos conceptos.

\begin{figure}[!htb]
\centering
\minipage{0.65\textwidth}
  \includegraphics[width=\linewidth]{adc}
  \caption{Convertidor analógico-digital. La señal digital resultante forma palabras con  bits, es decir, se tienen en total $2^3=8$  valores diferentes y únicos}\label{fig:adc}
\endminipage\hfill
\end{figure}

Llevando esta idea al campo de los sistemas digitales, tal y como puede ser un ordenador, se llega al uso de la lógica de dos estados o binaria en la cual existen los estados alto H o 1 y bajo L o 0 en el caso de lógica positiva y H o 0 junto a L o 1 para la lógica negativa.
Un ejemplo de señal digital para la lógica de dos estados:

\begin{figure}[!htb]
\centering
\minipage{0.65\textwidth}
  \includegraphics[width=\linewidth]{digital_signal}
  \caption{Ejemplo de comunicación maestro escalvo mediante señales digitales.}\label{fig:digital_signal}
\endminipage\hfill
\end{figure}

\end{itemize}

%https://difiere.com/la-diferencia-analogo-digital/

Volviendo al enfrentamiento entre el funcionamiento de la visión humana y artificial, se establece la comparación de ambas a continuación:

\begin{itemize}
\item Ojo humano
\\
Por una parte el ser humano recibe información visual tal y como se describiría en un mundo analógico,es decir, de forma continua.
\\
Sin entrar en demasiado detalle en el ámbito anatómico, la visión en el hombre se explica como la capacidad del ojo para detectar la luz y transformar la energía lumínica en señales eléctricas las cuales viajan al cerebro mediante el nervio óptico. Entre sus componentes principales se encuentran la córnea, la parte más externa del ojo, el cristalino, una lente ajustable según la distancia al objetivo así como un "diafragma" denominado pupila, cuyo diámetro está regulado por el iris, y la retina que se trata del tejido sensible a la luz. 

El funcionamiento del ojo se explica porque la luz es refractada por la córnea al atravesarla. Esta luz refractada pasa a través de la pupila y el cristalino y se proyecta sobre la retina, zona en la que unas células fotorreceptoras la transforman en impulsos nerviosos que se trasladan, a través del nervio óptico, al cerebro.

%https://steemit.com/science/@greenrun/the-human-eye-and-vision-a-fascinating-phenomenon

\begin{figure}[!htb]
\centering
\minipage{0.50\textwidth}
  \includegraphics[width=\linewidth]{human_vision}
  \caption{Esquema del proceso de adquisición de información visual a través del ojo.}\label{fig:human_vision}
\endminipage\hfill
\end{figure}

Se ve claramente un tipo de señal analógica presente en el proceso, la luz, o mejor dicho la intensidad de la misma que pude representarse por la luminancia, es decir, candela por metro cuadrado.

%https://es.wikipedia.org/wiki/Intensidad_luminosa
%https://es.wikipedia.org/wiki/Ojo_humano#Examen_del_ojo

\item Sensor artificial
\\
No se puede aplicar de forma directa el concepto del funcionamiento del ojo humano en un sensor artificial ya que trabaja con información digitalizada. 

De este modo hace falta unos pasos intermedios antes de llegar a un resultado final Por lo tanto hay que utilizar una aproximación diferente para resolver el problema y es aquí donde entra el concepto de nube de puntos. 
\end{itemize}

\section{Concepto de nube de puntos} \label{section:nubes_ejemplo}

Una nube de puntos se puede definir como una estructura $P$ que representa un conjunto de puntos multidimensionales $p \subset R_{n}$. En el caso de una nube de puntos en tres dimensiones, es decir $n=3$, cada elemento o punto está representado como mínimo por sus coordenadas geométricas $X,Y, Z$ respecto a un sistema de referencia dado. Pero se puede añadir más datos todavía en forma de color, curvatura o información sobre la normal $\vec{n}$ a una superficie en un ámbito local de la misma.  


Por lo tanto una nube de puntos es un conjunto de puntos individuales sin relación alguna entre ellos, cuya
posición, color y otro tipo de características tienen definición, edición y representación muy simple por lo
que es realmente práctico y sencillo manejar una gran cantidad de ellos sin tener que preocuparse por
conceptos como escala, rotaciones y demás relaciones entre diferentes puntos de un mismo objeto, siempre y cuando se tome la consideración más básica de la idea de nube de puntos.

%https://www.3deling.com/whta-is-a-point-cloud/
%https://www.3deling.com/rgb-point-cloud/

Se puede derivar de este concepto una gran potencia y flexibilidad ya que si se tienen una cantidad
suficiente de puntos dispuestos correctamente se pueden representar todo tipo de superficies aunque en
realidad no se trate de un plano continuo, es decir, el cerebro es capaz de interpretar complejas formas a
partir de un tipo de información tan sencilla como las tres coordenadas espaciales.

Es más, se pueden llevar a cabo conversiones para relacionar el conjunto de puntos de la nube y crear
superficies reales en tres dimensiones. Este tipo de conversión se denomina también como reconstrucción de superficies, es decir, se parte de información puntual y se crea una superficie continua estimando qué
relación puede haber entre puntos cercanos. De esta forma, una nube de puntos puede transformarse a una
malla de polígonos o triángulos o incluso modelos CAD.

Las técnicas de reconstrucción de superficies son variadas y entre ellas se encuentran la triangulación de
delaunay, que construye una red de triángulos sobre los vértices de la nube de puntos.

Sobre la triangulación de delaunay, se debe cumplir la condición que toma el mismo nombre sobre la nube de puntos en la que se quiere reconstruir la superficie y la cual establece que:
\\
"la circunferencia circunscrita de un triángulo no debe contener ningún otro vértice de la triangulación en su interior, admitiéndose vértices situados sobre la circunferencia"
\\
En este contexto se entiende que por "vértice" se indica un punto de la nube de puntos.
Por lo tanto una red de triángulos es una triangulación de Delaunay si todos y cada uno de los triángulos que la forman cumplen la condición descrita que puede aplicarse tanto en espacios bidimensionales como tridimensionales.

La apreciación gráfica del mencionado concepto puede verse a continuación:
\begin{figure}[!htb]
\minipage{0.32\textwidth}
  \includegraphics[width=\linewidth]{delaunay_mal}
  \caption{Vértice en el interior de una circunferencia circunscrita. No se cumple la condición de Delaunay}\label{fig:del_mal}
\endminipage\hfill
\minipage{0.32\textwidth}
  \includegraphics[width=\linewidth]{delaunay_bien}
  \caption{Vértice fuera de una circunferencia circunscrita. Se cumple la condición de Delaunay}\label{fig:del_bien}
\endminipage\hfill
\minipage{0.32\textwidth}
  \includegraphics[width=\linewidth]{delaunay_bien_10pts}
	\caption{Triangulación de Delaunay aplicada a 10 puntos. Ninguna de las circunferencias circunscritas contiene vértices en su interior.}\label{fig:del_bien_10pts}
\endminipage
\end{figure}
El origen del nombre de la condición de Delauany se debe al matemático ruso Boris Nikolaevich Delone quien lo ideó en 1934 y tomó la forma francesa de su apellido, "Delaunay" como referencia a sus antecesores franceses.

%https://en.wikipedia.org/wiki/Point_cloud

Por otra parte, una peculiaridad o limitación respecto a las nubes de puntos tiene que ver con que la
información que representan es superficial, es decir, los puntos siempre pertenecen a la superficie del
objeto en cuestión ya que es el lugar donde la luz de los escáneres se llega, rebota y devuelve la
información correspondiente, proceso que se explica con más profundidad en apartados posteriores.

Otra desventaja inherente a las nubes de puntos es la interpretación de la información que contienen ya
que se ha explicado que está compuesta de un conjunto de objetos o puntos sin relación entre ellos. Es
aquí donde se requiere intervención del ser humano pues es su cerebro el que puede encontrar la similitud
entre una nube de puntos dada y el objeto o escenario que se supone que representa. Existe software capaz
de encontrar patrones y características para clasificar nubes de puntos pero nunca de forma
completamente fiable.

El concepto de nube de puntos es eminentemente simple así como versátil y de gran utilidad. Esto se puede apreciar con gran multitud de aplicaciones del concepto de nube de puntos en el mundo real y es que este progreso tecnológico es un gran paso adelante para refinar procesos ya existentes, desde producción a nivel industrial hasta establecer las bases de la navegación de cualquier robot o vehículo.

Se muestran a continuación varios ejemplos de diferentes características:

En primer lugar se tiene en la figura \ref{fig:bunny_simple} una sencilla nube de puntos que representa un conejo. Se ha tomado una captura desde uno de todos los posibles puntos de vista que ofrece una visión en 360º.
Un ejemplo algo más elaborado se recoge en la figura \ref{fig:wolf} en la que aparece una nube que representa un lobo. 
Otro ejemplo de mayor complejidad permite observar en la figura \ref{fig:botes} un escaneo frontal de tres botes de plástico. En este caso, es obvio que las zonas vacías justo detrás de los botes son aquellos lugares donde los rayos de luz emitidos por el sensor no pueden llegar ya que los propios botes los bloquean lo cual sirve para que su superficie quede capturada. Además, se ha incluido un campo de color para cada punto.


\begin{figure}[!htb]
\minipage{0.32\textwidth}
  \includegraphics[width=\linewidth]{bunny_simple}
  \caption{Nube de puntos representando un conejo.
  Peso total de la nube: 10.6KB.
  Número total de puntos: 397.}\label{fig:bunny_simple}
\endminipage\hfill
\minipage{0.32\textwidth}
  \includegraphics[width=\linewidth]{wolf}
  \caption{Nube de puntos representando un lobo.
  Peso total de la nube: 42.6KB.
  Número total de puntos: 3400.}\label{fig:wolf}
\endminipage\hfill
\minipage{0.32\textwidth}%
  \includegraphics[width=\linewidth]{botes}
  \caption{Nube de puntos representando tres botes.
  Peso total de la nube: 2.43MB.
  Número total de puntos: 307200.}\label{fig:botes}
\endminipage
\end{figure}


%http://www.pointclouds.org/news/2013/01/07/point-cloud-data-sets/
%http://graphics.stanford.edu/data/3Dscanrep/
%http://kos.informatik.uni-osnabrueck.de/3Dscans/
%https://www.cc.gatech.edu/~turk/bunny/bunny.html

Una vez vistos ejemplos de nubes de puntos con la información suficiente para reconocer qué objeto representan sin más ayuda que la de los propios ojos, es momento de subir el nivel de complejidad para dar lugar a nubes de puntos como las que se muestran a continuación:

Se ha visto anteriormente una sencilla representación de un conejo con solamente 397 puntos. El nivel de detalle pude incrementarse a niveles del orden de decenas de miles de puntos tal y como es el caso de la figura \ref{fig:bunny} que representa un conejo de arcilla de 7.5 pulgadas de alto con unos 69451 triángulos ya que se ha llevado a cabo la reconstrucción de la superficie. Además, en la figura \ref{fig:bunny_colored} se aprecia el resultado si se añade información sobre color en cada punto.
Como consecuencia de añadir más información (más de 90 veces la cantidad de puntos y el color) se tiene en este caso un peso de 22MB con hasta 35947 puntos.
\begin{figure}[!htb]
\minipage{0.45\textwidth}
  \includegraphics[width=\linewidth]{bunny}
  \caption{Nube de puntos con reconstrucción de superficie representando un conejo sin información de color.}\label{fig:bunny}
\endminipage\hfill
\minipage{0.45\textwidth}
  \includegraphics[width=\linewidth]{bunny_colored}
  \caption{Nube de puntos con reconstrucción de superficie representando un conejo con información de color.}\label{fig:bunny_colored}
\endminipage\hfill
\end{figure}
Esta nube de puntos proviene del departamento de computación gráfica de la universidad de Stanford e hicieron falta un total de 10 escaneos con el escaner Cyberware 3030 MS para llegar al resultado final. Para hacerse una idea del tipo de sensor utilizado, se tiene como dato relevante su precio de unos 10000 dólares (ebay) teniendo en cuenta además de que se trata de un sensor antiguo pues el escaneo se produjo en 1993.

%https://www.ebay.com/itm/Cyberware-Rapid-3D-Digitizer-Model-Shop-Color-3030-Scanner-3030RGB-MS-/372338340673?_ul=AR

Se pueden representar objetos más grandes y con mayor nivel de detalle tal y como se aprecia en la figura \ref{fig:dragon} que representa un dragón construido con madera y resina y con un tamaño de aproximadamente 20cm x 8cm x 9cm.

\begin{figure}
\centering
\includegraphics[scale=0.3]{dragon}
\caption{Nube de puntos con reconstrucción de superficie representando una figura de un dragón.}\label{fig:dragon}
\end{figure}

En este caso se han llevado a cabo 18 escaneos con una resolución de 100$\mu$m o lo que es lo mismo, la separación entre puntos es del orden de 0,1mm. Se dispone de un total de 3609455 puntos y 7218906 triángulos lo que implica un peso de 86MB para la nube reconstruida y descomprimida.
La nube de puntos se generó en el mismo laboratorio y con el mismo escaner que se ha mencionado en el caso anterior.

Otro objeto de elevada complejidad que ha sido escaneado en las mismas condiciones que el dragón y el conejo es el ángel Lucy. Un total de 47 escaneos dan lugar a un resultado final de 14027932 puntos y 28055742 triángulos y para este caso un peso de 508MB tomando la nube de puntos reconstruida y descomprimida.

\begin{figure}
\centering
\includegraphics[scale=0.27]{angel_lucy}
\caption{Nube de puntos con reconstrucción de superficie representando una figura de un ángel.}\label{fig:angel_lucy}
\end{figure}

No solamente se pueden representar objetos mediante nubes de puntos sino entornos abiertos o interiores. Johannes Schauer y Andreas Nüchter de la universidad de Würzburg, Alemania, tomaron una nube de puntos del mercado en la ciudad de Würzburg tal y como se ve en la figura \ref{fig:wue_city}
El escaner utilizado en el este caso es el Riegl VZ-400 y con un total de 6 escaneos se han conseguido 86585411 puntos conteniendo cada uno de ellos información sobre la reflectancia de la luz del sensor lo que se aprecia con puntos de diferente claridad ya que no hay información sobre color. Obviamente, un entorno exterior contiene mucha más información que un simple objeto por lo que esta nube de puntos tiene un peso de 5117MB descomprimida.

Cabe destacar que el sensor utilizado es bastante más potente que el anteriormente mencionado y tiene un precio de unos 80000 dólares.

%http://www.riegl.com/uploads/tx_pxpriegldownloads/10_DataSheet_VZ-400_2017-06-14.pdf
%https://www.ebay.pl/itm/Riegl-VZ-400-3D-Terrestrial-Laser-Scanner-/331738609542

\begin{figure}
\centering
\includegraphics[scale=0.17]{wue_city2}
\caption{Nube de puntos con información de reflectancia de la luz que representa un mercado en Würzburg, Alemania.}\label{fig:wue_city}
\end{figure}

Por último, Dorit Borrmann obtuvo una nube de puntos que representa el interior del laboratorio de automática en la universidad de Jacobs, Bremen. Se pueden apreciar diferentes tipos de puntos para cada sección de la imagen ya que el sensor láser Riegl VZ-400 se encarga de representar información térmica y de profundidad mientras que las cámaras Optris PI IR y Logitech QuickCam 9000 Pro muestran información relacionada con el color.

\begin{figure}
\centering
\includegraphics[scale=0.17]{joinedmodel}
\caption{Unión de cuatro nubes de puntos con información térmica, de color y reflectancia de la luz representando un entorno cerrado.}\label{fig:joined_model}
\end{figure}

Se hicieron un total de 9 escaneos en 40º cada uno lo que permite disponer de una imagen en 360º (la imagen mostrada es uno de los escaneos)

Se han visto varios ejemplos que muestran la potencia del concepto de nube de puntos y es ahora el momento de conocer cómo se ha transformado una porción de la realidad en un conjunto de puntos que la representan.
\section{Adquisición de información: Sensores y evolución}

La adquisición y almacenamiento de información es el primer paso para comenzar a trabajar con una nube de puntos. A pesar de tratarse de información relativamente sencilla como coordenadas respecto a un sistema de referencia, color y reflectividad, hay que tomar dicha información para miles o incluso millones de puntos tal y como se ha visto en ejemplos mostrados anteriormente como en el caso del ángel Lucy (figura ). Por lo tanto, el factor limitante del sensor en cuestión radica en cuánta y cuan variada información es capaz de percibir y almacenar.

En los últimos veinte años, se han hecho grandes progresos en lo que a sensores se refiere ya que actualmente se usan de sofisticadas cámaras y escaneres láser y se han dejado atrás los sensores basados en sonar o infrarrojos los cuales proporcionan a penas unos bytes de información sobre el entorno u objeto que tratan de representar.

\subsection{Sensores láser}
Centrándose en los sensores láser, la adquisición de información cobra sentido cuando se estudia el comportamiento de los rayos de luz ya sea visible o infrarroja, por ejemplo. Para poder llevar a cabo el escaneo de objetos tridimensionales así como entornos, se usa el escaneo laser o también conocido como lidar (light detection and ranging pero originalmente se conocía por la union de light and radar), 
procedimiento que se originó a principios de la década de los sesenta tras la invención del láser y que permite medir distancias a un objetivo iluminándolo con pulsos láser.


Pero ¿Cómo se pueden medir distancias utilizando luz? el concepto entorno al que el escaner laser gira es el tiempo de vuelo. Esto quiere decir que se utiliza un dispositivo (range finder) capaz de medir con precisión el tiempo que transcurre desde que se emite un pulso de luz hasta que vuelve otra vez al mismo tras rebotar sobre el objeto que desea detectarse. Considerando entonces que la velocidad de la luz es una constante conocida, c, la distancia del escaner a un punto en concreto donde rebota un determinado pulso de luz puede determinarse como:

\begin{equation}
d = \frac{c*t}{2}
\end{equation}


Nótese que $c*t$ es la distancia que hay entre el escaner y el objeto duplicada ya que $t$ es el tiempo total desde la emisión del pulso de luz hasta la recepción. Por tanto, tomando la mitad del tiempo total de vuelo se obtiene la medida deseada que es la distancia del escaner al objeto.

\begin{figure}
\centering
\minipage{0.6\textwidth}
  \includegraphics[width=\linewidth]{lidar_explanation}
  \caption{Ejemplo ilustrativo de medición de distancia de un sensor lidar a un edificio}\label{fig:lidar explanation}
\endminipage\hfill

\end{figure}

Para cada pulso de luz emitido se detecta un punto en concreto lo que hace pensar que para poder crear nubes con millones de puntos la velocidad de generación de los pulsos ha de ser elevada. En el caso de lidar se pueden emitir hasta 150000 pulsos en un segundo.

De este modo, al ser capaz de medir la distancia que hay desde el punto de emisión de los rayos de luz hasta la superficie en la que rebotan, el sensor puede detectar rápidamente formas definidas de objetos, edificios o paisajes considerando en conjunto de puntos detectados tal y como se ha mostrado en ejemplos del apartado \ref{section:nubes_ejemplo}

\subsubsection{Características y control de la luz utilizada por el sensor}
El componente esencial en un sistema lidar es el rayo de luz que permite hacer las mediciones. Los láseres con que tienen entre 600 y 1000 nm de longitud de onda no suelen usarse para fines científicos y debido a que pueden ser fácilmente absorbidos por el ojo deben tener potencia limitada para que su uso sea seguro.
Por otra parte, los láseres con 1550nm de longitud de onda son una buena alternativa a los anteriores ya que no son focalizados por el ojo lo que los hace seguros con potencias mucho más elevadas. Este tipo de longitudes de onda se utiliza para aplicaciones a largo alcance que no requieren elevada precisión. 

Por lo general, hay dos tipos de lidar:
método coherente e incoherente o también conocido como detección de energía directa.
El método incoherente mide cambios en la amplitud de la onda emitida pues al rebotar e interactuar con el ambiente su nivel de energía varía.
El método coherente es más apropiado para medir diferencias en la frecuencia de la onda y utiliza modulación en fase y/o frecuencia de la misma lo que le permite operar a potencias mucho más bajas a costa de utilizar un equipamiento mucho más complejo.


En ambos modelos se pueden usar dos tipos diferentes de pulsos: micropulsos y sistemas de alta energía.
los micropulsos surgen de la elevada capacidad computacional de las computadoras actuales. Esto deriva en un láser de baja potencia (del orden del microjulio) que es clasificado como seguro al ojo permitiéndose su uso bajo escasas medidas de precaución 

Por otra parte, los sistemas de alta energía, requieren medidas de seguridad más estrictas y se usan principalmente para fines de investigación atmosférica pues permite tomar medidas como la altura, número de capas y densidad de las nubes, propiedades de las partículas en las nubes, temperatura, presión, concentración de gases, o humedad.

En cuanto al control del láser, en la mayoría de los escaneres, la dirección de emisión del rayo de luz es constante por lo que para poder apuntar el haz hacia la dirección deseada se hace uso de espejos. Esto implica lanzar el láser contra un espejo y controlar éste con diferentes movimientos de rotación; en un eje para un movimiento unidimensional (un grado de libertad) o en dos ejes para un alcance espacial total quedando fija la posición del foco emisor(dos grados de libertad)

\begin{figure}[!htb]
\minipage{0.32\textwidth}
  \includegraphics[width=\linewidth]{sensor_laser_espejo}
  \caption{Láser proyectado contra un espejo con un grado de libertad}\label{fig:sensor laser completo - espejo}
\endminipage\hfill
\minipage{0.32\textwidth}
  \includegraphics[width=\linewidth]{sensor_laser_vista_tope}
  \caption{Vista superior del sensor y las superficies que obstaculizan el haz de luz: objeto y recinto en el que se encuentra}\label{fig:sensor laser completo - vista tope}
\endminipage\hfill
\minipage{0.32\textwidth}
  \includegraphics[width=\linewidth]{sensor_laser_resultado}
  \caption{Conjunto de puntos obtenidos tras el esacaneo pertenecientes al objeto en el interior del recinto y las limitaciones espaciales del mismo.}\label{fig:sensor laser completo - resultado}
\endminipage
\end{figure}

Una alternativa para controlar el rayo láser en dos dimensiones incumbe el uso de dos espejos montados en ejes ortogonales y entonces hacer movimientos de rotación entorno a un eje para cada uno de los espejos.

\begin{figure}
\minipage{0.45\textwidth}
  \includegraphics[width=\linewidth]{sensor_dos_espejos}
  \caption{Uso de dos espejos con un grado de libertad en cada uno y accionados con galvanómetros}\label{fig:sensor dos espejos}
\endminipage\hfill
\minipage{0.45\textwidth}
  \includegraphics[width=\linewidth]{movimiento_lidar_2D}
  \caption{Sensor lidar con dos grados de libertad.}\label{fig:movimiento lidar 2D}
\endminipage\hfill

\end{figure}


para sistemas más sofisticados todavía, se requiere el posicionamiento espacial del foco emisor de rayos lo cual se consigue con un sistema de lentes servo controladas conocido como focus shifter o z-shifter

La forma más común de mover los espejos es mediante el uso de un motor eléctrico o un galvanómetro para el más sencillo de los casos. Se pueden utilizar también actuadores piezoeléctricos o magnetorresistivos para una mayor velocidad angular pero a costa de menores ángulos máximos de desplazamiento

\subsubsection{Ventajas y desventajas de lidar}

Retomando la idea de que la velocidad de la luz es una constante, la única variable en el cálculo de la distancia es el tiempo de vuelo. Téngase por ejemplo una distancia de 30cm desde el sensor hasta el objeto que desea capturarse. Esto implica que la resolución del reloj integrado en el sensor ha de ser cuanto menos elevada:

\begin{equation}
\frac{0.3m}{3*10^{8}\frac{m}{s}} = 1ns =0,000000001s
\end{equation}

Se ha desvelado de esta forma una desventaja del concepto de tiempo de vuelo, se necesita equipamiento muy preciso y fiable lo que se traduce en complejidad y elevadas inversiones económicas.

Pero dando la vuelta a esta desventaja, es decir, cuando se trata de escanear objetos en la lejanía como puede ser un edificio o paisaje, la resolución requerida por parte del reloj se reduce dando así medidas más fiables. Además, considerando de nuevo la velocidad de la luz como una constante, no importa la distancia a la que se encuentre el objeto salvo por cuestiones de difracción y absorción del pulso de luz en el ambiente, por ejemplo, por la presencia de humedad.

Por lo tanto, el método lidar combina precisión y versatilidad ya que puede valerse de luz visible, infrarroja o ultravioleta para lanzarla contra objetivos de diversos tipos de materiales como metal, cerámica, aerosoles, terreno (rocas y tierra) e incluso se puede llegar al nivel molecular. 
 
Sin embargo no se realiza un único escaneo ya que el propio concepto implica que el objeto bloquea los rayos de luz por lo que la cara frontal, de la que sí se obtiene información, impide a la luz llegar a la cara posterior. El factor clave es entonces el hecho de que estos rayos de luz puedan llegar a toda la superficie del objeto que se quiere analizar, es decir, accesibilidad física del sensor.
De este modo, independientemente del sensor o método que se utilice, es imposible recolectar
información sobre superficies no visibles o lo que es lo mismo, con un solo escaneo.

Este efecto puede apreciarse en la figura \ref{fig:botes} donde falta información en el seno de la nube de puntos en forma de tres áreas en las que no hay ningún punto pues son la "sombra" de los botes ante los rayos de luz del escaner.

Como consecuencia, es necesario llevar a cabo varios escaneos desde diferentes puntos de vista y
ponerlos en conjunto conociendo con precisión la posición del sensor en cada escaneo para poder llevar a cabo lo que se conoce como alineamiento de nubes de puntos, concepto que se explicará con más detalle posteriormente.


\subsubsection{Aplicaciones de lidar}

Una aplicación muy extendida del lidar es el reconocimiento de terreno. Para ello se integra el sensor en una aeronave y se capturan los puntos correspondientes al terreno sobrevolado. Esta aplicación es útil para generar modelos digitales de elevación.

Como contrapartida, hay que tener en cuenta que el sensor tiene una posición variable respecto al terreno ya que va embarcado en una aeronave. Para considerar esto en el resultado final de la nube de puntos generada, se ha de disponer de un sistema de navegación, para el posicionamiento del sensor con GPS por ejemplo, así como de una unidad de medidas inerciales (IMU inertial maesurement unit), para obtener información sobre la orientación absoluta del sensor.

 
\begin{figure}
\minipage{0.45\textwidth}
  \includegraphics[width=\linewidth]{airborne_lidar}
  \caption{Esquema de utilización del método lidar en una aeroanve}\label{fig:airborne_lidar}
\endminipage\hfill
\minipage{0.45\textwidth}
  \includegraphics[width=\linewidth]{airborne_city}
  \caption{Entorno urbano recreado con el método lidar}\label{fig:airborne_city}
\endminipage\hfill

\end{figure}




%https://upload.wikimedia.org/wikipedia/commons/c/c0/LIDAR-scanned-SICK-LMS-animation.gif
%https://www.engineering.com/AdvancedManufacturing/ArticleID/12390/Quality-Basics-How-Does-3D-Laser-Scanning-Work.aspx
%https://en.wikipedia.org/wiki/Laser_scanning
%http://www.2grobotics.com/wp-content/uploads/2017/03/sonarvslaser.pdf
%http://www.lidar-uk.com/how-lidar-works/
%http://elm-chan.org/works/vlp/report_e.html
%http://www.ionix.fi/en/technologies/laser-processing/laser-marking/
%https://blog.cometlabs.io/engineer-explains-lidar-748f9ba0c404

\subsection{Alternativas a lidar}

Disponer de complejas, fiables y robustas representaciones del mundo real tiene no es tan sencillo como pueda parecer puesto que estos sensores suelen tener un precio prohibitivo para la mayoría de los interesados ya sean particulares o incluso empresas con un poder adquisitivo considerable. Véanse por ejemplo los sensores mencionados en el apartado \ref{section:nubes_ejemplo} los cuales tienen un precio de más de 70000\$

Sin embargo, la situación ha cambiado desde que han aparecido en el mercado ciertos sensores 3D como por ejemplo el sensor Kinect de la consola Xbox360 de Microsoft. Este sensor está basado en la tecnología PrimeSense y aunque puede trabajar con nubes de puntos en tiempo real e imágenes en 2D su precio no supera los 150\$. De este modo se ha producido un gran paso adelante en cuanto a los impedimentos relacionados con la adquisición, mantenimiento y delicadeza del hardware que traduce el mundo real a nubes de puntos.

%https://www.matec-conferences.org/articles/matecconf/pdf/2018/32/matecconf_smima2018_03001.pdf
%https://erget.wordpress.com/2014/04/27/producing-3d-point-clouds-with-a-stereo-camera-in-opencv/

\section{Procesamiento software de información } \label{librerias}

Una vez registradas y almacenadas las nubes de puntos, se precisa ahora de un mecanismo para trabajar con la inmensa cantidad de información que aportan los sensores. El software existente para dicha tarea es diverso y no siempre gratuito. Como ejemplo se tiene RealityCapture o RC, un software capaz de crear modelos 3D a partir de fotografías o escaneos láser desordenados. Su alcance abarca aspectos como arte, arquitectura, escaneo completo del cuerpo humano, videojuegos, mapeado, efectos visuales y realidad virtual.

Entre sus características se encuentran cálculo de redes de polígonos, coloreado, texturizado, georreferenciación, conversión de sistema de coordenadas, suavizado y operaciones de lectura/escritura o input/output.
El software puede ejecutarse por línea de comandos o mediante un kit de desarrollo.

A pesar de que puede llegar a mezclar imágenes de cámara y escaneos láser, este software está diseñado para desarrollar bajas exigencias de hardware. 

Trabaja de forma lineal lo que significa que si las entradas se duplican también lo hará el tiempo de ejecución.

Este software está disponible solamente en inglés y sus requerimientos son su punto débil ya que necesita una máquina de 64 bits con al menos 8 GB de RAM y una versión de windows superior a windows 7. Además, se puede disponer de forma opcional de una GPU nVidia CUDA de al menos 1GB de RAM si se desea crear redes texturizadas.

Cada licencia está limitada a 32 núcleos de procesador y 3 tarjetas gráficas. Para configuraciones superiores, se pueden adquirir más licencias. Se recomienda una computadora con un procesador de 4 núcleos, 16GB de RAM y una CUDA 386.



Existe más software semejante al descrito y la inmensa mayoría tiene limitaciones de licencias o de hardware demasiado potente, y por lo tanto de un valor económico elevado, para un usuario particular. Pero entre todos ellos destaca PCL, siglas que en inglés se refieren a Point Cloud Library, es decir, librería de nubes de puntos escrita en lenguaje C++. Es una librería única, de gran escala y un proyecto abierto a la comunidad para procesamiento de imágenes y nubes de puntos tanto en 2D como en 3D. Es más, PCL se somete a os términos de la licencia BSD (Berkelay Software Distribution) lo que implica que es de uso libre tanto para fines comerciales como de investigación. 

\begin{figure}
\centering
\minipage{0.45\textwidth}
  \includegraphics[width=\linewidth]{pcl_logo}
  \caption{Logo de PCL}\label{fig:pcl logo}
\endminipage\hfill
\end{figure}

PCL dispone de multitud de herramientas para el procesamiento de nubes de puntos y dado que es una librería sometida a constante crecimiento y modificaciones, tener las herramientas debidamente organizadas y estructuradas es de vital importancia. Es por esto por lo que PCL tiene una organización en forma de librerías modulares, es decir, el conjunto de herramientas se clasifica según su utilidad.

Dichas librerías se mencionan a continuación con una breve descripción de su utilidad:

\begin{table}
\begin{tabular}{|L|c|L|}\hline
 \textbf{Nombre} & \textbf{Definición} \\\hline 
 
filters &  \multicolumn{1}{m{10cm}|}{Contiene mecanismos de eliminación de ruido y filtrado de puntos con determinadas características}\\\hline

  features  & \multicolumn{1}{m{10cm}|}{Contiene estructuras de datos y mecanismos para estimación de características en 3D. Permiten detectar patrones geométricos haciendo uso de la información local que representa el entorno de un punto dentro de una nube}\\\hline
  
  keypoints  & \multicolumn{1}{m{10cm}|}{Contiene las implementaciones de dos algoritmos de detección de keypoints o puntos clave los cuales se explicarán detalladamente más adelante}\\\hline
  
  registration  & \multicolumn{1}{m{10cm}|}{Permite unificar diferentes conjuntos de datos (nubes de puntos) en un modelo global y consistente. Se vale de los keypoints de cada uno de los conjuntos iniciales para poder desarrollar el modelo final unificado}\\\hline
  
  kdtree  & \multicolumn{1}{m{10cm}|}{Proporciona estructuras de datos que particionan el espacio de trabajo para almacenar puntos en una estructura de árbol y así mejorar la eficiencia de las operaciones realizadas sobre la nube de puntos}\\\hline
  
  octree  & \multicolumn{1}{m{10cm}|}{Porporciona métodos para crear estructuras de datos jerárquicas en forma de árbol para particionar el espacio de trabajo y agilizar futuras operaciones sobre la nube}\\\hline
  
  segmentation  & \multicolumn{1}{m{10cm}|}{Contiene algoritmos para segmentar o dividir una nube en diferentes regiones. Su utilidad se revela al trabajar con nubes que constan de grupos aislados de puntos}\\\hline
  
sample consensus  & \multicolumn{1}{m{10cm}|}{Contiene modelos de figuras geométricas predefinidas como líneas, planos o cilindros para detectar similitudes respecto a los mismos en una nube de puntos dada. Estos modelos se pueden combinar para dar lugar a una gran variedad de figuras geométricas}\\\hline

  surface  & \multicolumn{1}{m{10cm}|}{Permite la reconstrucción de superficies sobre escaneos en 3D}\\\hline
  
  recognition  & \multicolumn{1}{m{10cm}|}{Contiene algoritmos para el reconocimiento de diferentes tipos de objetos}\\\hline
  
  io  & \multicolumn{1}{m{10cm}|}{Contiene clases y métodos para leer y escribir nubes de puntos en formato PCD que se explicará más adelante}\\\hline
  
   visualization  & \multicolumn{1}{m{10cm}|}{Sirve para visualizar nubes de puntos y así apreciar los resultados de las operaciones realizadas sobre las mismas}\\\hline
 
\end{tabular}
\end{table}

Abordar en profundidad todos y cada uno de estos módulos conllevaría un trabajo arduo y tedioso de modo que únicamente se entrará en detalle en aquellos que sean de utilidad para el propósito del presente trabajo y a lo cual se procede en el siguiente capítulo.

A parte de los módulos o agrupaciones dentro de la librería de PCL, son necesarias otras herramientas que vienen implementadas de forma independiente en librerías externas:

\begin{itemize}
\item[•]Eigen: Es una librería escrita en C++ para operaciones de álgebra lineal como operaciones con vectores, matrices o métodos numéricos.
\item[•]FLANN: Son la siglas de Fast Library for Approximate Nearest Neighbors que se traduce como librería de aproximación rápida de vecinos más cercanos.Está escrita en C++ y sirve para realizar búsquedas rápidas de puntos vecinos cercanos a uno en concreto en espacios de diferentes dimensiones.
\item[•]boost: Se trata de un conjunto de bibliotecas para extender las capacidades del lenguaje de programación C++ 
\item[•]VTK: Son las siglas de Visualization Toolkit o kit de visualización. Está escrita en C++ y permite procesamiento y visualización de imágenes así como aporta soporte para multitud de algoritmos de visualización como es el escalar, vectorial, y métodos más avanzados como la triangulación de Delaunay
\end{itemize}










\chapter{Herramientas empleadas para la realización del proyecto}

\section{Introducción}
Habiendo tratado en el capítulo anterior el fundamento teórico que concierne a las nubes de puntos y el procesamiento software de las mismas se procede en este capítulo a explicar con mayor profundidad las herramientas necesarias para cumplir los objetivos propuestos para el presente trabajo.



\section{Descripción de herramientas para desarrollar el trabajo}
Para realizar el presente trabajo se requieren herramientas que pueden clasificarse en dos tipos: software y hardware.

\subsection{Herramientas software}
\subsubsection{Máquina virtual}
Se hace uso de una máquina virtual con Ubuntu\cite{ubuntu} 18.04.1 64 bits ya que es un sistema operativo que facilita la instalación y uso de PCL y otras librerías, no como Windows.
Por lo tanto, a partir de este punto, salvo que se mencione lo contrario, el sistema de archivos e instrucciones ejecutadas por línea de comandos así como demás características propias de diferentes sistemas operativos se refieren a un sistema operativo basado en Linux.
\\
\\
La máquina virtual se crea haciendo uso del software VirtualBox\cite{virtualbox} el cual facilita la creación y personalización de máquinas virtuales no solamente basadas en Linux sino con cualquier otro sistema operativo.

%https://www.virtualbox.org/
\subsubsection{Instalación de PCL sobre un sistema basado en Linux}
Una vez se dispone de la máquina virtual, es necesario instalar las librerías de PCL\cite{pcl_installation}. Para ello se visita la web oficial de PCL pues ésta ofrece las instrucciones adecuadas. Para poder instalar PCL en linux se deben ejecutar los siguientes comandos:

%http://pointclouds.org/downloads/

\begin{verbatim}
sudo add-apt-repository ppa:v-launchpad-jochen-sprickerhof-de/pcl
sudo apt-get update
sudo apt-get install libpcl-all
\end{verbatim}

La primera instrucción añade al sistema el repositorio en el que se encuentra la librería, el segundo busca actualizaciones disponibles y por último se procede a la instalación de todos los archivos actualizados.
\\
\\
Estas mismas instrucciones sirven para instalar PCL en el sistema embebido ya que dispone del un sistema operativo Debian basado en Linux.
\\
\\
Cuando la instalación está completa, se generan una serie de carpetas en el directorio /usr/include:

\begin{itemize}
\item[•]pcl-1.8: Es la versión 1.8 de la librería de PCL. Contiene en su interior la carpeta pcl que con todos los módulos de PCL estructurados correctamente.
\item[•]eigen3: Librería eigen que se encuentra bajo la carpeta Eigen dentro de este directorio.
\item[•]FLANN: Librería FLANN que se encuentra bajo la carpeta flann dentro de este directorio.
\item[•]vtk-x: versión x de la librería vtk 
\end{itemize}

\subsubsection{Vivado Design Suite HLx Editions}
Ya está instalado PCL en el sistema, ahora falta adquirir una herramienta para síntesis de alto nivel.
\\
\\
La síntesis de alto nivel, del inglés High Level Synthesis (HLS) es un proceso de diseño automático que interpreta una descripción algorítmica en software de un comportamiento deseado y crea hardware digital que lo implementa.
\\
\\
La síntesis comienza con una descripción en alto nivel del problema o comportamiento que desea reproducirse. Para ello se puede utilizar uno de los varios lenguajes de programación en alto nivel como C o C++. Este código es entonces analizado y programado para ser compilado en lo que se conoce como un Register Transfer Level (RTL), es decir, la abstracción del diseño que modela un circuito digital síncrono en sus señales entre registros y las operaciones realizadas sobre ellas.  El diseño de hardware puede ser generado a diferentes niveles de abstracción: puertas lógicas, registros y algoritmos. Se aprecia en la figura \ref{fig:circuito} un diseño hardware a nivel de registro de un circuito inversor.
\begin{figure}[!htb]
\centering
\includegraphics[scale=0.25]{circuito}
  \caption{Circuito digital síncrono que actúa como un inversor de la señal de entrada.}\label{fig:circuito}
\end{figure}

El RTL queda definido mediante un Hardware Description Language (HDL) del inglés, lenguaje de descripción de hardware como VHDL o Verilog. Para el caso del ejemplo anterior y tomando VHDL como el lenguaje de descripción, la definición del circuito de la figura \ref{fig:circuito} queda como:

\begin{lstlisting}[language=VHDL,breaklines]
D <= not Q;
 
process(clk)
begin
    if rising_edge(clk) then
        Q <= D;
    end if;
end process;
\end{lstlisting}


El objetivo de la síntesis en alto nivel es permitir a los diseñadores construir y verificar hardware de forma eficiente así como el de dotarles de mayor control y optimización sobre sus diseños con la facilidad añadida de poder definirlos mediante lenguajes de alto nivel de abstracción mientras la herramienta realiza la implementación del RTL de forma automática.
\\
\\
Para realizar la síntesis de alto nivel se elige la herramienta vivado HLS de Xilinx. Accediendo a la web oficial de descargas\cite{vivado_descarga} se puede descargar la versión deseada de este software ya sea como un archivo comprimido un instalador web para mayor comodidad. En este caso se elige la versión 2017.1
\\
\\
Haciendo uso del mencionado software de Xilinx, se define en un lenguaje de programación de alto nivel, por ejemplo C++, un comportamiento que se implementa como una función con sus argumentos, operaciones internas y un valor de retorno. Además, deben definirse unos ``pragmas'', es decir, una región de código considerada como un protocolo y en el que Vivado HLS no introduce ninguna señal de reloj excepto si se indica explícitamente. Una región definida como un protocolo puede ser utilizada para especificar de forma manual una interfaz de conexión a otros bloques hardware con el mismo protocolo de entrada/salida de señales. Esto implica que, por ejemplo, dado un bloque hardware, se pueden definir dos buses: un bus IN que contiene todas las señales de entrada al bloque y un bus OUT que contiene todas las señales salientes.


%https://www.xilinx.com/support/download.html
%https://www.xilinx.com/products/design-tools/vivado/integration/esl-design.html

\subsection{Herramientas hardware} \label{herraminetas_hardware}
La principal herramienta hardware de la que se dispone y sobre la cual se comprobarán los resultados de la aplicación de los objetivos del presente trabajo es la placa de desarrollo Pynq-Z1\cite{pynq} que se puede ver en la figura \ref{fig:pynq}. Se trata de una placa de bajo coste y ampliamente utilizada en universidades y centros de investigación.

\begin{figure}[H]
\centering
\includegraphics[scale=1.0]{pynq}
  \caption{Placa de desarrollo Pynq-Z1 de Xilinx.}\label{fig:pynq}
\end{figure}

Esta placa dispone de un System on Programmable Chip (SoPC) del inglés sistema en chip programable, del modelo Zynq xc7z020clg400-1. Un sistema en chip programable es la combinación de núcleos de procesamiento de una CPU (Central Processing Unit o unidad central de procesamiento) con hardware personalizado que se implementa normalmente con una FPGA y bloques de memoria.
\\
\\
Una FPGA, siglas del inglés ``Field Programmable Gate Array'' o matriz de puertas programables es un dispositivo programable que contiene bloques de lógica cuya interconexión y funcionalidad pueden ser reconfiguradas en tiempo de ejecución.
\\
Además, una FPGA dispone de dos capas: aplicación y configuración. La primera se compone de todos sus recursos hardware que han de ser configurados e interconectados haciendo uso de la segunda, la capa de configuración en la que se carga un ``bitstream'' es decir un archivo de configuración que se puede generar, entre otras formas, haciendo uso del programa Vivado de Xilinx y que permite para cada ``bitstream'' que se cargue, dotar a la FPGA de diferentes funcionalidades.
\\
\\
Volviendo al ámbito de un SoPC, los núcleos del procesador pueden ser ``hard'' (duros) o ``soft'' (blandos): los primeros son permanentemente embebidos en silicio mientras que los segundos se implementan usando recursos de una FPGA. 
\\
Los núcleos duros ofrecen mucho rendimiento y bajo consumo pero son poco flexibles en cuanto a las operaciones que pueden realizar. Éstos forman el Processing System (PS) dentro del SoPC para rutinas software y operaciones relacionadas con sistemas operativos a diferencia de la FPGA que conforma la Programmable Logic (PL) que sirve para implementar lógica de alta velocidad y aritmética.
\\
La PS dispone en este caso de un sistema operativo Debian basado en el kernel Linux.
\\
\\
%http://ati.ttu.ee/~alsu/05_SoPC.pdf
Algunas de las características principales del PS del SoPC mencionado son:

\begin{itemize}
\item[•] Un procesador Cortex-A9 de dos núcleos a 650MHz y arquitectura ARM, memoria caché de nivel 1 con 32KB pra instrucciones y 32KB para datos, memoria caché de nivel 2 y 512KB y memoria en chip de 256KB.
\item[•] Controlador de memoria DDR3 con 8 canales DMA y 4 puertos esclavos AXI3 de alto rendimiento
\item[•] Controladores de periféricos con elevado ancho de banda: Ethernet 1G, USB 2.0 y SDIO 
\item[•] Controlador de periféricos de gran ancho de banda: SPI, UART, CAN, I2C
\item[•] 630KB de memoria RAM
\item[•] Programable con JTAG, flash Quad-SPI y tarjeta micro SD
\end{itemize}


En cuanto al PL del SoPC:
\begin{itemize}
\item[•] 13300 secciones de lógica cada una de ellas con 4 Look-up Tables de 6 entradas y 8 Flip-Flops
\item[•]630KB de block RAM
\item[•] 4 unidades de administración de la señal de reloj
\item[•] 220 secciones de lógica para procesamiento de señales digitales (DSP)
\item[•] Conversor analógico-digital integrado en chip (XADC)

\end{itemize}

\section{Conclusiones}
Con el cierre del presente capítulo terminan las explicaciones fundamentales tanto de teoría como de herramientas que permiten estructurar el TFG.
\\
En el siguiente capítulo se mostrarán las primeras tareas que giran entorno a la librería PCL para así poder visualizar nubes de puntos y extraer keypoints de las mismas.


\chapter{Reproducción de algoritmos de PCL: Visualización de nubes y extracción de keypoints}

Puesto que el código escrito en cualquier lenguaje de programación puede resultar complicado de transmitir, la forma de proceder para conseguir dicho objetivo será la siguiente; En primer lugar se explicará en alto nivel el programa en cuestión sin necesidad de leer código y haciendo uso de flujogramas u otros métodos que se consideren adecuados. Después, habiendo entendido la funcionalidad del programa, se pasará a explicar el código que lo compone con el nivel de detalle adecuado en cada momento. Para esto, también aparecerán flujogramas así como la explicación de las distintas partes del programa por parte del autor.  

\section{Visualización de nubes de puntos}
Tal y como se ha mencionado en el apartado de objetivos del presente trabajo, la visualización de nubes se considera un hito transversal pero que a la vez involucra todo el proyecto ya que es la parte más amigable al ojo humano para estudiar resultados. De este modo se aprovecha el libre uso de la documentación y los tutoriales ofrecidos por PCL para modificar el código a favor de los objetivos planteados en el TFG.
Se recuerda que la visualización de nubes no puede llevarse a cabo en la FPGA sobre la que se implementan los objetivos principales del trabajo ya que ésta no dispone de interfaz gráfica.
%http://pointclouds.org/documentation/tutorials/

Para la visualización básica de nubes de puntos se necesita en primer lugar hacer uso del módulo IO que permite leer nubes en formato PCD. Cuando la nube está cargada, utilizando el módulo visualization, se crea una nueva ventana que hace de visualizador y en la que aparecen los tres ejes coordenados XYZ y el conjunto de puntos que conforman la nube situados en el espacio respecto al origen de los mencionados ejes. Cuando el usuario lo desee, puede cerrar la ventana que el programa ha creado para terminarlo.
Este conjunto de operaciones se muestran en su explicación en alto nivel en forma de flujograma en la figura \ref{fig:visualization_diagram}

\begin{figure}
\centering
\includegraphics[scale=0.5]{visualization_diagram}
\caption{Flujograma del proceso de visualización de nubes de puntos.}\label{fig:visualization_diagram}
\end{figure}

El funcionamiento de este programa es bastante sencillo. Véase a continuación el código que hace posible la funcionalidad ya explicada.


\begin{lstlisting}[language=C++,breaklines]
#include <iostream>
#include <pcl/visualization/cloud_viewer.h>
#include <pcl/io/io.h>
#include <pcl/io/pcd_io.h>
#include <pcl/console/parse.h>
    
void 
viewerOneOff (pcl::visualization::PCLVisualizer& viewer)
{
    viewer.setBackgroundColor (1.0, 0.5, 0.5);
    pcl::PointXYZ o;
    o.x = 0;
    o.y = 0;
    o.z = 0;
    viewer.addSphere (o, 0.01, "sphere", 0); 
}
    
int 
main (int argc, char** argv)
{   

    pcl::PointCloud<pcl::PointXYZRGBA>::Ptr cloud (new pcl::PointCloud<pcl::PointXYZRGBA>);
    std::vector<int> pcd_filename_indices = pcl::console::parse_file_extension_argument (argc, argv, "pcd"); 
    std::string filename;
     
    if (!pcd_filename_indices.empty ())
    {
    	filename = argv[pcd_filename_indices[0]];
    	if (pcl::io::loadPCDFile (filename, *cloud) == -1)
    	{
      		cerr << "Was not able to open file \""<<filename<<"\".\n";
      		return -1;
    	}
    }
    else
    {
    	cout << "\nNo *.pcd file given => closing.\n\n";
    	return -1;
    }
    
    cout << "\nNumber of points in "<< filename << ": " << cloud->points.size() << "\n";
        
    pcl::visualization::CloudViewer viewer("Cloud Viewer");
    viewer.showCloud(cloud);
    
    viewer.runOnVisualizationThreadOnce (viewerOneOff);
    
    while (!viewer.wasStopped ())
    {
    }
    return 0;
}
\end{lstlisting}


En primer lugar, se cargan las librerías de iostream para las funcionalidades básicas de C++ y determinados módulos de la librería PCL; $IO$ par ala lectura de nubes, $cloud_viewer$ del módulo visualization para la visualización de nubes y $parse$ del módulo console para poder evaluar los argumentos introducidos por la consola de comandos.

\begin{lstlisting}[language=C++,breaklines]
#include <iostream>
#include <pcl/visualization/cloud_viewer.h>
#include <pcl/io/io.h>
#include <pcl/io/pcd_io.h>
#include <pcl/console/parse.h>
\end{lstlisting}


En el siguiente fragmento de código se implementa la función que se ejecutará una sola vez cuando se abre la ventana que visualiza la nube de puntos. Su funcionalidad, tal y como adelanta el flujograma ya expuesto, se basa en añadir una esfera en el origen de coordenadas para localizarlo fácilmente. Se trata del objeto $o$ de la clase $PointXYZ$ creado en la línea . Además, en la línea  se configura el color de fondo del visualizador aportando pesos a los valores máximos de color rojo, verde y azul que se pueden tener; 1.0 para el color rojo y 0.5 para los colores verde y azul. Finalmente, con el método $addSphere$ se añade al visualizador a esfera creada.
\begin{lstlisting}[language=C++,breaklines]
void 
viewerOneOff (pcl::visualization::PCLVisualizer& viewer)
{
    viewer.setBackgroundColor (1.0, 0.5, 0.5);
    pcl::PointXYZ o;
    o.x = 0;
    o.y = 0;
    o.z = 0;
    viewer.addSphere (o, 0.01, "sphere", 0); 
}
\end{lstlisting}


A continuación, dentro del método principal $main$, se crea un objeto de nube de puntos llamado $cloud$ que contiene información de coordenadas XYZ y color RGBA, refiriéndose RGB a Red, Green y Blue respectivamente, es decir, rojo, verde y azul. la letra $A$ se refiere a la intensidad de color. 
También se crea el objeto $pcd_filename_indices$ que guarda el índice del argumento pasado por consola de comandos que contiene la nube de puntos en formato PCD. En $filename$ se guarda el nombre de la nube de puntos.
Si $pcd_filename_indices$ está vacío quiere decir que el usuario no ha introducido ninguna nube de puntos en formato PCD por lo que el programa termina en este punto con el mensaje de error correspondiente. En caso contrario, se procede a leer y almacenar la nube de puntos de entrada y si el método de lectura falla, el programa termina con el mensaje de error correspondiente.
Si todo va bien y se consigue leer la nube y almacenarla en el objeto $cloud$, a modo informativo, se muestra por pantalla el número de puntos que contiene la nube accediendo al atributo SIZE de la misma (referencia a formato PCD)


\begin{lstlisting}[language=C++,breaklines]
int 
main (int argc, char** argv)
{   

    pcl::PointCloud<pcl::PointXYZRGBA>::Ptr cloud (new pcl::PointCloud<pcl::PointXYZRGBA>);
    std::vector<int> pcd_filename_indices = pcl::console::parse_file_extension_argument (argc, argv, "pcd"); 
    std::string filename;
     
    if (!pcd_filename_indices.empty ())
    {
    	filename = argv[pcd_filename_indices[0]];
    	if (pcl::io::loadPCDFile (filename, *cloud) == -1)
    	{
      		cerr << "Was not able to open file \""<<filename<<"\".\n";
      		return -1;
    	}
    }
    else
    {
    	cout << "\nNo *.pcd file given => closing.\n\n";
    	return -1;
    }
    
    cout << "\nNumber of points in "<< filename << ": " << cloud->points.size() << "\n";
\end{lstlisting}

Por último, se crea un objeto de visualización llamado $viewer$ y el cual se utiliza para mostrar la nube con la llamada a $showcloud$ y asignar el método $runOnVisualizationThreadOnce$ explicado previamente para ejecutarse al comienzo de la visualización de la nube. 
Siempre y cuando el proceso de visualización no haya sido detenido, la nube seguirá mostrándose por la ventana creada para este propósito. Esto se consigue con el bucle $while$

\begin{lstlisting}[language=C++,breaklines]
	pcl::visualization::CloudViewer viewer("Cloud Viewer");
    viewer.showCloud(cloud);
    viewer.runOnVisualizationThreadOnce (viewerOneOff);
    
    while (!viewer.wasStopped ())
    {
    }
    return 0;
}
\end{lstlisting}

el otro vsiualizadooooooooooooooooooor

el programa principaaaaaaaaaaaaaaaaaaaaaaal

\begin{lstlisting}[language=C++,breaklines]

// STL
#include <iostream>

// PCL
#include <pcl/io/pcd_io.h>
#include <pcl/io/impl/pcd_io.hpp>

#include <pcl/point_types.h>

#include <pcl/common/io.h>

#include <pcl/keypoints/sift_keypoint.h>
#include <pcl/keypoints/narf_keypoint.h>
#include <pcl/features/normal_3d.h>
#include <pcl/features/impl/normal_3d.hpp>

#include <pcl/impl/pcl_base.hpp>

#include <pcl/search/pcl_search.h>
#include <pcl/search/impl/search.hpp>
#include <pcl/search/impl/organized.hpp>
#include <pcl/search/impl/kdtree.hpp>

#include <pcl/filters/impl/voxel_grid.hpp>

#include <pcl/kdtree/impl/kdtree_flann.hpp>

#include <ctime>

#include <boost/thread/thread.hpp>

#include <pcl/features/normal_3d_omp.h>

#include <pcl/console/parse.h>

#include <fstream>

int normal_estimation_object = 0;
float radius_search = 0.02f;
float normal_estimation_time = 0.0f;
float sift_estimation_time = 0.0f;

float min_scale = 0.01f;
int n_octaves = 3;
int n_scales_per_octave = 4;
float min_contrast = 0.001f;
int sift_points=0; 

clock_t begin,end;
double elapsed_sec;

void 
printUsage (const char* progName)
{
  std::cout << "\n\nUsage: "<<progName<<" [options] <scene.pcd>\n\n"
            << "Options:\n"
            << "-------------------------------------------\n"
            << "-o <integer>	0 for regular normal estimation (default), 1 for enhanced normal estimation\n"
            << "-r <float>	Radius search for normal estimation (default "<< radius_search<<")\n"
            << "-ms <float>	Minimum scale (default " << min_scale << ")\n"
            << "-no <int>	Number of octaves (default " << n_octaves << ")\n"
            << "-ns <int>	Number of scales per octave (default " << n_scales_per_octave << ")\n"
	    << "-mc <float>	Minimum contrast (default " << min_contrast << ")\n"
	    << "-h		Show help\n"
            << "\n\n";
}

int main(int argc, char** argv)
{

  if(argc == 1 || (pcl::console::find_argument (argc,argv,"-h") >= 0) )
  {
	printUsage (argv[0]);
	return 0;
  }	

  std::cout << std::endl << "---Normal estimation parameters---" << std::endl;

  pcl::NormalEstimation<pcl::PointXYZ, pcl::PointNormal> ne;
  pcl::console::parse (argc, argv, "-o", normal_estimation_object);
  if(normal_estimation_object >0)
  {
	std::cout << "Using enhanced normal estimation object" << std::endl;
	pcl::NormalEstimationOMP<pcl::PointXYZ, pcl::PointNormal> ne;
  }
  else
  {
	std::cout << "Using regular normal estimation object" << std::endl;
  }
  
  pcl::console::parse (argc,argv, "-r", radius_search);
  std::cout << "Setting radius search for normal estimation to: " << radius_search << std::endl;

  
  std::cout << std::endl << "---Sift points parameters---" << std::endl;
  
  
  pcl::console::parse (argc,argv, "-ms", min_scale);
  std::cout << "Setting minimum scale to: " << min_scale << std::endl;

  pcl::console::parse (argc,argv, "-no", n_octaves);
  std::cout << "Setting number of octaves to: " << n_octaves << std::endl;

  pcl::console::parse (argc,argv, "-ns", n_scales_per_octave);
  std::cout << "Setting number of scales per octave to: " << n_scales_per_octave << std::endl;

  pcl::console::parse (argc,argv, "-mc", min_contrast);
  std::cout << "Setting minimum contrast to: " << min_contrast << std::endl;

  std::cout << std::endl << std::endl;

  begin = clock();

  pcl::PointCloud<pcl::PointXYZ>::Ptr cloud_xyz (new pcl::PointCloud<pcl::PointXYZ>);
  std::vector<int> pcd_filename_indices = pcl::console::parse_file_extension_argument (argc, argv, "pcd"); 
  std::string filename;
     
  std::cout << "Reading file..." << std::endl;

  if (!pcd_filename_indices.empty ())
  {
  	filename = argv[pcd_filename_indices[0]];
  	if (pcl::io::loadPCDFile (filename, *cloud_xyz) == -1) 
    	{
        	std::cout << "Was not able to open file \""<<filename<<"\".\n";
       		return -1;
    	}
  }
  else
  {
  	std::cout << "\nNo *.pcd file given => closing.\n\n";
  	return -1;
  }
  
  end = clock();
  elapsed_sec = double(end-begin)/CLOCKS_PER_SEC;
  std::cout << "Number of points in "<< filename << ": "<< cloud_xyz->points.size () <<std::endl; 
  std::cout << "Time needed for " << filename << " to load: " << elapsed_sec << " seconds"<< std::endl << std::endl; 
 
  
  pcl::PointCloud<pcl::PointNormal>::Ptr cloud_normals (new 		pcl::PointCloud<pcl::PointNormal>);
  pcl::search::KdTree<pcl::PointXYZ>::Ptr tree_n(new pcl::search::KdTree<pcl::PointXYZ>());

  ne.setInputCloud(cloud_xyz);
  ne.setSearchMethod(tree_n);
  ne.setRadiusSearch(radius_search);
 
  std::cout << "Estimating normals in " << filename << " surface..." <<std::endl;

  begin = clock();
  ne.compute(*cloud_normals);
  end = clock();

  normal_estimation_time = double(end-begin)/CLOCKS_PER_SEC;
  std::cout << "Time needed for normal estimation (compute) in " << filename << ": " << normal_estimation_time << " seconds" << std::endl << std::endl;

//----Copy the xyz info from cloud_xyz and add it to cloud_normals as the xyz field in PointNormals estimation is zero---
  
  std::cout << "Copying xyz information from" << filename << " to cloud with normals information..." << std::endl;

  begin = clock();

  for(size_t i = 0; i<cloud_normals->points.size(); ++i)
  {
  	cloud_normals->points[i].x = cloud_xyz->points[i].x;
  	cloud_normals->points[i].y = cloud_xyz->points[i].y;
  	cloud_normals->points[i].z = cloud_xyz->points[i].z;
  }

  end = clock();
  elapsed_sec = double(end-begin)/CLOCKS_PER_SEC;
  std::cout << "Time needed for copying the pointcloud: " << elapsed_sec <<" seconds" << std::endl << std::endl;
  if(cloud_normals->points.size()!=0){
  	pcl::io::savePCDFileASCII ("cloud_normals.pcd", *cloud_normals);
  }

  pcl::SIFTKeypoint<pcl::PointNormal, pcl::PointWithScale> sift;
  pcl::PointCloud<pcl::PointWithScale>::Ptr result(new pcl::PointCloud<pcl::PointWithScale>);
  pcl::search::KdTree<pcl::PointNormal>::Ptr tree(new pcl::search::KdTree<pcl::PointNormal> ());
  sift.setSearchMethod(tree);
  sift.setScales(min_scale, n_octaves, n_scales_per_octave);
  sift.setMinimumContrast(min_contrast);
  sift.setInputCloud(cloud_normals);
 
  std::cout << "Estimating sift points in " << filename << "..." << std::endl;

  begin = clock();
  sift.compute(*result);
  end = clock();
  sift_estimation_time = double(end-begin)/CLOCKS_PER_SEC;
  std::cout << "Time needed for sift point extraction: " << sift_estimation_time << " seconds" << std::endl << std::endl;


  if(result->points.size()>0){
  
  	std::cout << "Number of SIFT points in " << filename << ": " << result->points.size () << std::endl;

	sift_points = result->points.size();

	pcl::PointCloud<pcl::PointXYZRGBA>::Ptr keypoints(new pcl::PointCloud<pcl::PointXYZRGBA>);
  
	keypoints->width = result->width;
	keypoints->height = result->height;
	keypoints->points.resize(keypoints->width * keypoints->height);

   	for (size_t i = 0; i < result->points.size (); ++i)
  	{
    		keypoints->points[i].x = result->points[i].x;
    		keypoints->points[i].y = result->points[i].y;
    		keypoints->points[i].z = result->points[i].z;

  		keypoints->points[i].r=50;
  		keypoints->points[i].g=255;
  		keypoints->points[i].b=50;
  		keypoints->points[i].a=255;
  	}
  	pcl::io::savePCDFileASCII ("sift_keypoints.pcd", *keypoints);
  }
  else {
  	std::cout << "No sift points found" << std::endl;
	sift_points = 0;
  }
 
  std::fstream fs;
  fs.open("tests.txt", std::fstream::app);
  
  fs << "filename: " << filename << std::endl;
  
  fs << std::endl <<  "Normal estimation radius search: " << radius_search << std::endl; 
  fs << "Minimum scale: " << min_scale << std::endl;
  fs << "Number of octaves: " << n_octaves << std::endl;
  fs << "Number of scales per octave: " << n_scales_per_octave << std::endl;
  fs << "Minimum contrast: " << min_contrast << std::endl;

  fs << std::endl << "Normal estimation time (s): " << normal_estimation_time << std::endl;
  fs << "SIFT points estimation time (s): " << sift_estimation_time << std::endl;
  fs << "Number of SIFT points found: " << sift_points << std::endl;

  fs << "---------------------\n----------------------\n";
  fs.close();
  return 0;
}
\end{lstlisting}

\chapter{Medición de tiempos de ejecución}

\section{Introducción}
Se ha mencionado en el capítulo anterior que el programa de extracción de puntos SIFT, $sift\_keypoints$, consta de dos partes principales entre otras como pueden ser cargar librerías, crear o modificar ficheros de texto o copiar información de una nube de puntos a otra. Estas dos partes son la extracción de vectores normales a la superficie descrita por la nube de puntos con la que trabaja el programa y la estimación de puntos SIFT.


En este capítulo se va a presentar una forma sencilla de medir los tiempos de ejecución de estos dos procesos a la vez que se modifican los parámetros que influyen en ellos. De este modo, se determinará si será la estimación de normales o la de puntos SIFT la parte que necesita ser llevada a hardware digital para su posterior optimización. Estas pruebas se van a realizar sobre la FPGA puesto que es en este hardware donde se desarrollarán los objetivos principales del TFG.

Además, se ha creado una sección exclusiva para los gráficos dada la gran cantidad de ellos que hay.


\section{Método de medición}
Trabajando en C++, como de costumbre, el proceso de medición de tiempos requiere la inclusión de la librería $ctime$ para lo cual se tiene al comienzo del código en $sift\_keypoints.cpp$.

\begin{lstlisting}[language=C++,breaklines]
	#include <ctime>
\end{lstlisting}

Incluída esta librería, ahora se pueden utilizar las variables globales $begin$,$end$ y $elapsed\_sec$ declaradas de la siguiente manera:

\begin{lstlisting}[language=C++,breaklines]
	clock_t begin,end;
	double elapsed_sec;
\end{lstlisting}

Las variables $begin$ y $end$ son del tipo $clock\_t$ y sirven para almacenar cuentas de $clock\;\; ticks$ o ciclos de reloj mientras que $elapsed\_sec$ es del tipo double para almacenar el tiempo en segundos que ha llevado efectuar una determinada operación.\\
El proceso de medición de tiempo es el siguiente: 

\begin{lstlisting}[language=C++,breaklines]
	begin = clock();
	process();
	end = clock();
	
	elapsed_sec = double(end - begin)/CLOCKS_PER_SEC;
\end{lstlisting}

En primer lugar se almacena en $begin$ el valor devuelto por la función $clock()$, es decir, número de ciclos de reloj que han transcurrido en el procesador desde el inicio del programa hasta su llamada. A continuación se ejecuta el proceso cuyo tiempo de ejecución desea medirse, para este ejemplo, $process()$. Ahora se actúa de manera similar al primer paso ya que se almacena en $end$ el número de ciclos transcurridos desde el inicio del programa hasta la llamada a $clock()$ después de haberse ejecutado $process()$. Por último, se toma la diferencia de ciclos entre $end$ y $begin$ para saber cuántos ciclos ha tomado la ejecución de $process()$. Sin embargo esto no es una medida directa de tiempo por lo que el número de ciclos de ejecución del proceso se divide entre una constante, el número de ciclos que duran un segundo y que se llama $CLOCKS\_PER\_SEC$. El resultado se almacena en $elapsed\_sec$ como el tiempo en segundos que ha durado la ejecución de $process()$.


\section{Resultados de medición sobre la FPGA}
Una vez visto cómo se mide el tiempo de ejecución de cualquier parte del programa $sift\_keypoints$, se rescata parte del código del mismo para indicar qué procesos van a ser sometidos a esta medición.


En la parte de estimación de normales se analiza el método $compute$ el cual realiza todas las operaciones pertinentes para almacenar en $cloud\_normals$ los vectores normales a partir de $cloud\_xyz$:


\begin{lstlisting}[language=C++,breaklines]
  pcl::PointCloud<pcl::PointNormal>::Ptr cloud_normals (new 		pcl::PointCloud<pcl::PointNormal>);
  pcl::search::KdTree<pcl::PointXYZ>::Ptr tree_n(new pcl::search::KdTree<pcl::PointXYZ>());

  ne.setInputCloud(cloud_xyz);
  ne.setSearchMethod(tree_n);
  ne.setRadiusSearch(radius_search);
 
  std::cout << "Estimating normals in " << filename << " surface..." <<std::endl;

  begin = clock();
  ne.compute(*cloud_normals);
  end = clock();

  normal_estimation_time = double(end-begin)/CLOCKS_PER_SEC;
  std::cout << "Time needed for normal estimation (compute) in " << filename << ": " << normal_estimation_time << " seconds" << std::endl << std::endl;
\end{lstlisting}

Se procede de forma análoga en la parte de estimación de puntos SIFT puesto que, tras establecer el valor de los parámetros también se llama a un método $compute$ para efectuar la extracción de keypoints:

\begin{lstlisting}[language=C++,breaklines]
  pcl::SIFTKeypoint<pcl::PointNormal, pcl::PointWithScale> sift;
  pcl::PointCloud<pcl::PointWithScale>::Ptr result(new pcl::PointCloud<pcl::PointWithScale>);
  pcl::search::KdTree<pcl::PointNormal>::Ptr tree(new pcl::search::KdTree<pcl::PointNormal> ());
  sift.setSearchMethod(tree);
  sift.setScales(min_scale, n_octaves, n_scales_per_octave);
  sift.setMinimumContrast(min_contrast);
  sift.setInputCloud(cloud_normals);
 
  std::cout << "Estimating sift points in " << filename << "..." << std::endl;

  begin = clock();
  sift.compute(*result);
  end = clock();
  sift_estimation_time = double(end-begin)/CLOCKS_PER_SEC;
  std::cout << "Time needed for sift point extraction: " << sift_estimation_time << " seconds" << std::endl << std::endl;
\end{lstlisting}

Los tiempos de ejecución del método $compute$ se almacenan en $normal\_estimation\_time$ y $sift\_estimation\_time$ para la parte de estimación de normales y puntos SIFT, respectivamente.
\\
Para aportar mayor amplitud a los resultados obtenidos, se van a realizar mediciones de los métodos $compute$ variando ligeramente los parámetros involucrados en la estimación de normales y puntos SIFT. Como se ha visto en los fragmentos de código mostrados, la extracción de normales varía con un único parámetro, $radius\_search$, mientras que el tiempo de estimación de keypoints varía con $min\_scale$, $n\_octaves$ y $n\_scales\_per\_octave$. El parámetro $min\_contrast$ no afecta al tiempo del proceso estudiado porque, como se ha visto anteriormente, indica la dureza de la criba de los puntos SIFT ya obtenidos.

Las nubes sobre las que se han realizado las pruebas son las mismas que las mostradas en el capítulo anterior: $bunny.pc$, $cturtle.pcd$ y $milk\_cartoon\_all\_small\_clorox.pcd$

Tras efectuar la medición de tiempos junto a la variación de parámetros, éstos se han plasmado en diferentes gráficas. En primer lugar, se analizan los tiempos de estimación de normales y puntos SIFT variando el parámetro $radius\_search$ puesto que la extracción de normales es la primera operación y afecta a las que vienen después. El resultado se aprecia en las figuras \ref{fig:grafico_radius_bunny}, \ref{fig:grafico_radius_cturtle} y \ref{fig:grafico_radius_milk} las cuales indican que en todos los casos un incremento del radio de búsqueda de normales a la superficie dispara el tiempo de computación mientras que el tiempo de extracción de puntos SIFT se mantiene entorno a un valor estable a pesar de que la cantidad de normales extraídas afecta a la estimación de keypoints. Resulta obvio este incremento del tiempo de computación ya que la información que hay que utilizar para estimar un vector normal se incrementa, es decir, se considera una porción de la superficie de la nube cada vez más grande.



Se muestra en las figuras \ref{fig:grafico_min_scale_bunny}, \ref{fig:grafico_min_scale_cturtle} y \ref{fig:grafico_min_scale_milk} cómo varían estos mismos tiempos modificando el parámetro $min\_scale$. Se tiene en esta ocasión algo más de variedad en los resultados obtenidos ya que en $cturtle.pcd$ y $bunny.pcd$ el tiempo de estimación de keypoints comienza siendo mayor que el de extracción de normales pero a medida que incrementa $min\_scale$ éste decrece considerablemente. Esto se debe a que la nube de entrada se convoluciona con un filtro Gaussiano cada vez más severo, es decir, un aumento de la desviación típica difumina en mayor medida la nube y hace que ésta pierda detalles finos por lo que los puntos SIFT se encuentran en esquinas o bordes muy marcados y nítidos que son capaces de persistir ante el filtro aplicado. Por otra parte, el tiempo de estimación de normales no varía ya que su único parámetro, $radius\_search$, a partir de ahora tomará el valor por defecto. En cuanto a nubes a nubes con mayor nivel de detalle y cantidad de información como en $milk\_cartoon\_all\_small\_clorox.pcd$, se puede ver que aumentar el parámetro $min\_scale$ disminuye el tiempo de estimación de puntos SIFT siendo éste ya notablemente menor que el de extracción de normales.




Ahora se modifica el parámetro $n\_octaves$ y se aprecia que para casos realistas (nubes que contienen una gran cantidad de puntos como las nubes $milk\_cartoon\_all\_small\_clorox.pcd$ y $cturtle.pcd$) como se indican en las figuras \ref{fig:grafico_n_octaves_cturtle} y \ref{fig:grafico_n_octaves_milk} el tiempo de estimación de normales es en todo caso superior al de extracción de puntos SIFT incluso aunque se afine esta búsqueda aportando  más octavas en las que estimar keypoints. No es así en una nube pequeña como $bunny.pcd$ lo cual se puede considerar un caso particular tal y como indica la figura \ref{fig:grafico_n_octaves_bunny}.



Por último, se estudia el tiempo de estimación de keypoints ante la variación del parámetro $n\_scales\_per\_octave$ apreciándose los resultados en las figuras \ref{fig:grafico_n_scales_bunny}, \ref{fig:grafico_n_scales_cturtle} y \ref{fig:grafico_n_scales_milk}. Se tiene de nuevo un tiempo de estimación de normales predominante ante la extracción de puntos SIFT para nubes con una cantidad considerable de puntos como $cturtle.pcd$ y $milk\_cartoon\_all\_small\_clorox.pcd$. En esta ocasión, incrementar el número de convoluciones de la nube por octava implica realizar más operaciones y tener en cuenta la misma nube bajo más filtros para extraer keypoints por lo que el tiempo implicado en este proceso aumenta.


\section{Resultados gráficos}

\begin{figure}[h!]
\centering
\includegraphics[scale=1]{grafico_radius_bunny}
\caption{Tiempos de estimación de normales y puntos SIFT en $bunny.pcd$ variando $radius\_search$.}\label{fig:grafico_radius_bunny}
\end{figure}

\begin{figure}[h!]
\centering
\includegraphics[scale=1]{grafico_radius_cturtle}
\caption{Tiempos de estimación de normales y puntos SIFT en $cturtle.pcd$ variando $radius\_search$.}\label{fig:grafico_radius_cturtle}
\end{figure}


\begin{figure}[h!]
\centering
\includegraphics[scale=1]{grafico_radius_milk}
\caption{Tiempos de estimación de normales y puntos SIFT en $milk\_cartoon\_all\_small\_clorox.pcd$ variando $radius\_search$.}\label{fig:grafico_radius_milk}
\end{figure}


\begin{figure}[h!]
\centering
\includegraphics[scale=1]{grafico_min_scale_bunny}
\caption{Tiempos de estimación de normales y puntos SIFT en $bunny.pcd$ variando $min\_scale$.}\label{fig:grafico_min_scale_bunny}
\end{figure}

\begin{figure}[h!]
\centering
\includegraphics[scale=1]{grafico_min_scale_cturtle}
\caption{Tiempos de estimación de normales y puntos SIFT en $cturtle.pcd$ variando $min\_scale$.}\label{fig:grafico_min_scale_cturtle}
\end{figure}


\begin{figure}[h!]
\centering
\includegraphics[scale=1]{grafico_min_scale_milk}
\caption{Tiempos de estimación de normales y puntos SIFT en $milk\_cartoon\_all\_small\_clorox.pcd$ variando $min\_scale$.}\label{fig:grafico_min_scale_milk}
\end{figure}


\begin{figure}[h!]
\centering
\includegraphics[scale=1]{grafico_n_octaves_bunny}
\caption{Tiempos de estimación de normales y puntos SIFT en $bunny.pcd$ variando $n\_octaves$.}\label{fig:grafico_n_octaves_bunny}
\end{figure}

\begin{figure}[h!]
\centering
\includegraphics[scale=1]{grafico_n_octaves_cturtle}
\caption{Tiempos de estimación de normales y puntos SIFT en $cturtle.pcd$ variando $n\_octaves$.}\label{fig:grafico_n_octaves_cturtle}
\end{figure}

\begin{figure}[h!]
\centering
\includegraphics[scale=1]{grafico_n_octaves_milk}
\caption{Tiempos de estimación de normales y puntos SIFT en $milk\_cartoon\_all\_small\_clorox.pcd$ variando $n\_octaves$.}\label{fig:grafico_n_octaves_milk}
\end{figure}


\begin{figure}[h!]
\centering
\includegraphics[scale=1]{grafico_n_scales_bunny}
\caption{Tiempos de estimación de normales y puntos SIFT en $bunny.pcd$ variando $n\_scales\_per\_octave$.}\label{fig:grafico_n_scales_bunny}
\end{figure}

\begin{figure}[h!]
\centering
\includegraphics[scale=1]{grafico_n_scales_cturtle}
\caption{Tiempos de estimación de normales y puntos SIFT en $cturtle.pcd$ variando $n\_scales\_per\_octave$.}\label{fig:grafico_n_scales_cturtle}
\end{figure}


\begin{figure}[h!]
\centering
\includegraphics[scale=1]{grafico_n_scales_milk}
\caption{Tiempos de estimación de normales y puntos SIFT en $milk\_cartoon\_all\_small\_clorox.pcd$ variando $n\_scales\_per\_octave$.}\label{fig:grafico_n_scales_milk}
\end{figure}



\section{Conclusiones}
Habiendo estudiado cómo varían los tiempos de estimación de normales y puntos SIFT con diferentes valores de los parámetros implicados se concluye que se va a desarrollar hardware digital para su posterior optimización sobre la parte de extracción de vectores normales a la superficie representada por la nube de puntos. Esto es así por dos motivos:

\begin{itemize}
\item[•]Para nubes de puntos de gran tamaño, el tiempo de estimación de normales predomina sobre el de extracción de puntos SIFT.
\item[•]La estimación de vectores normales solamente varía bajo la acción de un parámetro, $radius\_search$, mientras que la extracción de puntos SIFT es más flexible puesto que está sometida a tres parámetros: $min\_scale$, $n\_octaves$ y $n\_scales\_per\_octave$.
\end{itemize}

En el siguiente capítulo se estudiará a fondo el algoritmo de extracción de normales puesto que se ha elegido éste para ser optimizado tras su síntesis en hardware digital.

\chapter{Análisis del algoritmo de extracción de normales usando PCL}

\section{Introducción}
Se ha justificado en el capítulo anterior que será el algoritmo de extracción de vectores normales el que se llevará a hardware digital para ser acelerado. 
\\
\\
En este capítulo, se va a estudiar en profundidad dicho algoritmo exclusivamente en el ámbito de la librería PCL ya que ésta se sirve de librerías externas como son principalmente Eigen y boost. Este estudio es necesario para comprender cómo funciona el algoritmo y así poder realizar implementarlo en hardware digital.
\\
\\
Por lo tanto, se explicará tanto en alto como en bajo nivel cómo PCL estima normales a la superficie de una nube de puntos de un modo semejante al que se ha utilizado para explicar fragmentos de código en capítulos anteriores. Finalmente, se mostrará el código en C++ que será modificado y sintetizado en hardware en el siguiente capítulo, empleando para ello herramientas de síntesis de alto nivel.


\section{Técnica de estimación de vectores normales}

%http://mediatum.ub.tum.de/doc/800632/941254.pdf

Las normales a una superficie son una de sus características geométricas más importantes no solamente en lo que concierne a la estimación de keypoints sino a otras áreas de computación gráfica como puede ser determinar las fuentes de luz adecuadas para generar sombras y brillos u otros efectos visuales semejantes. Por esta razón, la estimación de normales es una importante característica de la librería PCL\cite{normal}.
\\
\\
Si se considera una superficie, normalmente es trivial estimar la dirección de la normal en un determinado punto como el vector perpendicular a la superficie en el mismo. Sin embargo, puesto que las nubes de puntos adquiridas por los sensores son un conjunto de puntos que representan una superficie, hay dos formas de proceder para estimar normales:

\begin{itemize}
\item[•]Reconstruir la superficie que representan los puntos utilizando técnicas de reconstrucción de superficies y entonces calcular los vectores a partir de la superficie reconstruida.
\item[•]Utilizar aproximaciones para estimar las normales directamente a partir de la nube de puntos.
\end{itemize}

Dada la complejidad que implica la primera opción, se va a proceder de ahora en adelante con la segunda, estimar vectores normales a partir de puntos que representan una superficie.
\\
\\
El problema de determinar un vector normal a una superficie en un punto de la misma se puede aproximar mediante una de las formas más simples y claras que se pueden plantear. Esto implica la estimación de la normal a un plano tangente a la superficie en el punto estudiado junto a $k$ puntos vecinos\cite{normales_extra}, siendo entonces $P^k$ el mencionado conjunto y un punto en particular $p_{i} \in P^{k}$ representado por sus coordenadas mediante $p_{i}=\left\lbrace p_{i_x},p_{i_y},p_{i_z} \right\rbrace$.
\\
El plano utilizado para aproximar la normal en un punto de la nube es representado por un punto $x$ y un vector normal $\vec{n}$ de modo que la distancia de un punto $p_{i} \in P^{k}$ al plano queda definida como:

$$d_i=(p_{i}-x)\vec{n}$$
\\
Además, $x$ se define como el centroide de $P^{k}$ de la siguiente manera:

$$x=\frac{1}{k}\sum_{i=1}^{k} p_i$$
\\
Considerando lo anterior, se toman valores de $x$ y $\vec{n}$ de forma que, resolviendo un problema de mínimos cuadrados, $d_i$ sea cero para utilizar la mejor aproximación posible del plano tangente a la superficie de la nube en $p_i$.
\\
\\
Finalmente, la solución para $\vec{n}$ se da estudiando los autovectores y autovalores de la matriz de covarianzas $C \in {\rm I\!R} 3x3$ de $P^{k}$ calculada como:

$$C=\frac{1}{k}\sum_{i=1}^{k} (p_i-\bar{p})(p_i-\bar{p})^T,\;\;Cv_{j}=\lambda_{j}v_{j},\;\;j \in \left\lbrace 0,1,2 \right\rbrace$$
\\
$C$ es simétrica y semidefinida positiva con autovalores $\lambda_j \in {\rm I\!R} $ y autovectores $\vec{v_j}$.
\\
\\
Los autovectores son ortogonales entre sí y dan una aproximación de las principales componentes de $P^k$. No solo eso sino que si se cumple que:
 
$$0\leq\lambda_0\leq\lambda_1\leq\lambda_2$$
\\
entonces el autovector $\vec{v_0}$, el cual se corresponde con el autovalor de menor valor $\lambda_0$, es una aproximación de $\vec{n}= \left\lbrace n_x,n_y,n_z\right\rbrace$, vector normal a la superficie en el punto de la nube estudiado $p_i \in P^k$.
\\
\\
Debido a que no hay una forma estricta de determinar el signo del vector normal, su orientación es ambigua tras ser calculada con los procedimientos indicados anteriormente. Esto significa que las normales estimadas en la superficie de una nube de puntos no están consistentemente orientadas. Este efecto puede visualizarse en el ejemplo de la figura \ref{fig:normales_mal}.
\\
La solución para este problema es sencilla si se conoce la posición desde la cual se ha adquirido la nube, es decir, la posición del sensor. Para orientar todas las normales $\vec{n_i}$ hacia el punto de vista del sensor $v_p$ se debe cumplir:
$$\vec{n_i}(v_p-p_i)>0$$
Si se aplica esta corrección a la nube de la figura \ref{fig:normales_mal} se obtiene una orientación consistente de las normales tal y como se aprecia en la figura \ref{fig:normales_bien}.

\begin{figure}[!htb]
\minipage{0.48\textwidth}
  \includegraphics[width=\linewidth]{normales_mal}
  \caption{Vectores normales en una nube de puntos orientados de forma inconsistente.}\label{fig:normales_mal}
\endminipage\hfill
\minipage{0.48\textwidth}
  \includegraphics[width=\linewidth]{normales_bien}
  \caption{Vectores normales en una nube de puntos orientados de forma consistente.}\label{fig:normales_bien}
\endminipage\hfill
\end{figure}


\section{Estimación de normales a la superficie de una nube: alto nivel}
Conociendo el fundamento teórico relacionado con la estimación de normales a una superficie representada por un conjunto de puntos, se procede a continuación a explicar cómo PCL implementa este proceso.
\\
\\
Se retoma brevemente el código que permite calcular las normales en una nube de puntos:

\begin{lstlisting}[language=C++,breaklines]
  pcl::PointCloud<pcl::PointNormal>::Ptr cloud_normals (new 		pcl::PointCloud<pcl::PointNormal>);
  pcl::search::KdTree<pcl::PointXYZ>::Ptr tree_n(new pcl::search::KdTree<pcl::PointXYZ>());

  ne.setInputCloud(cloud_xyz);
  ne.setSearchMethod(tree_n);
  ne.setRadiusSearch(radius_search);
 
  std::cout << "Estimating normals in " << filename << " surface..." <<std::endl;

  begin = clock();
  ne.compute(*cloud_normals);
  end = clock();

  normal_estimation_time = double(end-begin)/CLOCKS_PER_SEC;
  std::cout << "Time needed for normal estimation (compute) in " << filename << ": " << normal_estimation_time << " seconds" << std::endl << std::endl;
\end{lstlisting}

Se recuerda que el método que lleva a cabo el algoritmo de estimación es $compute$ y cuyo contenido, explicado en alto nivel, es el que muestra el flujograma de la figura \ref{fig:compute_alto_nivel_diagram}.


\section{Estimación de normales a la superficie de una nube: bajo nivel}
\subsection{Consideraciones previas}
Para la explicación en bajo nivel (código en C++) de cómo PCL extrae las normales de una nube de puntos, se va a proceder de una forma mixta entre las explicaciones previas en alto y bajo nivel: se plantearán los flujogramas pertinentes cuyo contenido es la parte más relevante del código junto a las aclaraciones necesarias. Se procede de esta forma porque la estimación de normales no queda enteramente contenida al método $compute$ mencionado previamente, es decir, se hacen llamadas a otros métodos de la librería PCL o incluso de librerías externas y las cuales no se explicarán puesto que no son objetivo de aceleración.
\\
\\
Para llevar a cabo la explicación, se van a mostrar diferentes flujogramas conectados entre sí, cada uno con una funcionalidad y propósito propios, pero que en conjunto permiten reproducir el algoritmo de estimación de normales en una nube de puntos. Estos flujogramas se dividen en tres tipos: A, B y C que dividen la explicación en proceso de inicialización del algoritmo, estimación de normales y proceso de cierre del algoritmo, respectivamente.
\\
\\
Antes de proseguir con la explicación, se muestra en la figura \ref{fig:compute_esquema_clases} el esquema de los flujogramas que se usarán para la explicación y qué tipo de bloques cabe esperar de cada uno. Además, se indica las clases de PCL utilizadas y la relación de herencia entre ellas. Cabe destacar que:

\begin{itemize}
\item[•]\textit{PCLBase}\cite{pclbase} implementa los métodos utilizados por la mayoría de los algoritmos de toda la librería.
\item[•]\textit{Feature}\cite{feature} implementa métodos de estimación de características locales en una nube de puntos, como se ha explicado con anterioridad.
\item[•]\textit{NormalEstimation}\cite{normal_estimation} implementa métodos específicos de estimación de normales.
\end{itemize} 

Además, en los flujogramas, las partes del código en las que se centra la explicación está resaltada en color rojo.


\subsection{Cuerpo principal del algoritmo}
En la figura \ref{fig:compute_main} se puede ver el cuerpo del algoritmo de extracción de normales. El cuadro con un color naranja más intenso que los demás indica la llamada al método que desencadena todas las operaciones directamente relacionadas con la estimación de normales a parte de los procesos de inicialización y comprobaciones que efectúan los demás bloques. Tanto el desarrollo de la estimación de normales como los procesos auxiliares se implementan en flujogramas separados.

\subsubsection{Flujogramas tipo A: proceso de inicialización}
Este tipo de flujogramas se refieren al proceso de inicialización explicado en la figura \ref{fig:compute_main}. Este proceso está implementado en la librería de PCL pero no se explica ya que no es de interés para su aceleración puesto que implementa operaciones triviales y simples tales como asignaciones y comprobaciones que ocupan una pequeña parte del código.


%\begin{figure}[h!]
%\centering
%\includegraphics[scale=0.5]{compute_init}
%\caption{Flujogramas tipo A que indican el proceso de %inicialización del algoritmo de estimación de normales.}\label{fig:compute_init}
%\end{figure}


\subsubsection{Flujogramas tipo B: estimación de normales}

La figura \ref{fig:compute_compute_feature} muestra el flujograma principal de tipo B que explica el bucle que recorre todos los puntos de la nube y dentro del cual se llaman a tres métodos: $searchForNeighbors$, $computePointNormal$ y $flipNormalTowardsViewpoint$. Estos métodos se explican en sus respectivos flujogramas.
\\
\\
Las tres extensiones del flujograma B son B.1, B.2 y B.3 y explican, respectivamente, la búsqueda de vecinos entorno al punto estudiado, el cálculo de la normal en el punto estudiado y el proceso de cambiar de sentido la normal, si fuera necesario. No se muestran en detalle B.1 ni B.3 pues su contenido no está la librería de PCL para el caso del cálculo de vecinos y la operación de voltear un vector es bastante escueta y difícilmente acelerable. Por otra parte, se tiene el corazón del algoritmo en el flujograma B.2 que implementa el método \textit{computePointNormal}\cite{calculo_compute_point_normal} (en la figura \ref{fig:compute_computePointNormal}) y es el que llama a los métodos que explican el cálculo de la matriz de covarianzas (\textit{computeMeanAndCovarianceMatrix}) y la obtención del vector normal (\textit{solvePlaneParameters}) procesos que se explican en los flujogramas B.2.1 y B.2.2 respectivamente (figuras \ref{fig:compute_computeMean} y \ref{fig:compute_solvePlane})



%\begin{figure}[h!]
%\centering
%\includegraphics[scale=0.5]{compute_neighbors_flip}
%\caption{Flujogramas B.1 y B.3 para la estimación de vecinos y cambiar el sentido al vector normal}\label{fig:compute_neighbors_flip}
%\end{figure}

\subsubsection{Flujogramas tipo C: cierre del algoritmo}
Por último, se tiene un proceso de cierre del algoritmo que enlaza con la comprobación de la superficie de la nube sobre la que se van a buscar normales y que es realizada en la parte de inicialización. No se dan más detalles sobre este proceso ya que es realizado en software y no es objetivo de aceleración.


%\begin{figure}[h!]
%\centering
%\includegraphics[scale=0.6]{compute_deinit}
%\caption{Flujograma C que explica la última operación antes del %cierre del algoritmo.}\label{fig:compute_deinit}
%\end{figure}

\section{Selección del fragmento del algoritmo para su aceleración}
Se ha mencionado previamente que el núcleo del algoritmo la estimación de vectores normales a la superficie lo conforman el cálculo de matrices de covarianzas y las componentes de cada vector normal.
\\
\\
Se ha decidido seleccionar el cálculo de la matriz de covarianzas explicado en la figura \ref{fig:compute_computeMean} para su aceleración puesto que se reduce a operaciones con vectores y matrices que no requieren el uso de librerías específicas a diferencia del cálculo de las componentes del vector normal que necesita utilizar Eigen y da lugar a elementos no sintetizables en hardware. 

\section{Conclusiones}

Se ha explicado en este capítulo tanto a nivel teórico como práctico, es decir, haciendo uso de la librería PCL, la obtención de vectores normales a una superficie definida por una nube de puntos.
\\
En el siguiente capítulo, se mostrarán las modificaciones pertinentes al algoritmo de estimación de normales para que sea sintetizable en hardware y se explicará el proceso de aceleración llevado a cabo.

\section{Diagramas}

\begin{figure}[!htb]
\centering
\minipage{0.5\textwidth}
  \includegraphics[width=\linewidth]{compute_alto_nivel_diagram}
  \caption{Proceso simplificado de estimación de normales en una nube de puntos.}\label{fig:compute_alto_nivel_diagram}
\endminipage\hfill
\end{figure}


\begin{figure}[h!]
\centering
\includegraphics[scale=0.4]{compute_esquema_clases}
\caption{Esquema de bloques del flujograma para explicación en bajo nivel y relación entre las clases utilizadas de la librería PCL. En la figura, el flujograma principal se sitúa a la izquierda y hace una llamada a un flujograma A que se desarrolla en la parte derecha.}\label{fig:compute_esquema_clases}
\end{figure}

\begin{figure}[h!]
\centering
\includegraphics[scale=0.45]{compute_main}
\caption{Flujograma que representa el cuerpo del algoritmo de estimación de normales. Tiene conexión con otros tres flujogramas etiquetados como A, B y C.}\label{fig:compute_main}
\end{figure}

\begin{figure}[h!]
\centering
\includegraphics[scale=0.5]{compute_compute_feature}
\caption{Flujograma tipo B que agrupa la llamada a los métodos necesarios para la estimación de normales}\label{fig:compute_compute_feature}
\end{figure}

\begin{figure}[h!]
\centering
\includegraphics[scale=0.5]{compute_computePointNormal}
\caption{Flujograma B.2 que implementa la estimación de vectores normales.}\label{fig:compute_computePointNormal}
\end{figure}


\begin{figure}[h!]
\centering
\includegraphics[scale=0.5]{compute_computeMean}
\caption{Flujograma B.2.1 que explica el cálculo de la matriz de covarianzas.}\label{fig:compute_computeMean}
\end{figure}


\begin{figure}[h!]
\centering
\includegraphics[scale=0.5]{compute_solvePlane}
\caption{Flujograma B.2.2 que explica el cálculo de autovalores y autovectores para obtener el vector normal.}\label{fig:compute_solvePlane}
\end{figure}

\chapter{Modificación del código original y optimización mediante hardware digital}

\section{Introducción}
Se ha expuesto a grandes rasgos en el capítulo anterior cómo es el proceso de obtención de vectores normales a una superficie definida por una nube de puntos. Además, se ha indicado qué parte de dicho proceso será objeto de optimización con hardware digital, la obtención de matrices de covarianzas.

En el presente capítulo se exponen las diferencias entre el código original y el modificado para que sea sintetizable en hardware. Tras esto, se indicarán las optimizaciones realizadas sobre el código modificado y la generación de la IP (Intellectual Property) para el posterior trabajo con ella.


\section{Modificaciones en el código original}
En la librería PCL se pueden encontrar los siguientes fragmentos de código que permiten calcular la matriz de covarianzas asociada a un punto estudiado y sus vecinos dentro de la nube de puntos original sobre la que se desean obtener los vectores normales.

Como se ha explicado previamente, el método \textit{computeMeanAndCovarianceMatrix} recibe como argumentos la nube de puntos original como \textit{cloud} , un vector que contiene los índices de los puntos que forman una vecindad entorno al punto estudiado en la iteración actual como \textit{indices} y \textit{covariance\_matrix} y \textit{centroid} para guardar respectivamente la matriz de covarianzas y el centroide obtenidos.

\begin{lstlisting}[language=C++,breaklines]
  template <typename PointT, typename Scalar> inline unsigned int
pcl::computeMeanAndCovarianceMatrix (const pcl::PointCloud<PointT> &cloud,
                                     const std::vector<int> &indices,
                                     Eigen::Matrix<Scalar, 3, 3> &covariance_matrix,
                                     Eigen::Matrix<Scalar, 4, 1> &centroid)
{
\end{lstlisting}

Aparece la librería Eigen para crear una matriz llamada \textit{accu} en la que guardar resultados intermedios. Se comprueba si el atributo \textit{is\_dense} de la nube \textit{cloud} es \textit{true} o \textit{false} para llevar a cabo las mencionadas operaciones intermedias de diferentes maneras. En cualquiera de los casos, se recorre todo el vector de índices para obtener los resultados adecuados en \textit{accu} que posteriormente se divide entre el número de elementos del vector \textit{indices}.


\begin{lstlisting}[language=C++,breaklines]

  Eigen::Matrix<Scalar, 1, 9, Eigen::RowMajor> accu = Eigen::Matrix<Scalar, 1, 9, Eigen::RowMajor>::Zero ();
  size_t point_count;
  if (cloud.is_dense)
  {
    point_count = indices.size ();
    for (size_t i = 0; i <= point_count; ++i)
    {
      accu [0] += cloud[indices[i]].x * cloud[indices[i]].x;
      accu [1] += cloud[indices[i]].x * cloud[indices[i]].y;
      accu [2] += cloud[indices[i]].x * cloud[indices[i]].z;
      accu [3] += cloud[indices[i]].y * cloud[indices[i]].y;
      accu [4] += cloud[indices[i]].y * cloud[indices[i]].z;
      accu [5] += cloud[indices[i]].z * cloud[indices[i]].z;
      accu [6] += cloud[indices[i]].x;
      accu [7] += cloud[indices[i]].y;
      accu [8] += cloud[indices[i]].z;
    }
  }
  else
  {
    point_count = 0;
    for (size_t i = 0; i <= indices.end(); ++i)
    {
      if (!isFinite (cloud[indices[i]]))
        continue;

      ++point_count;
      accu [0] += cloud[indices[i]].x * cloud[indices[i]].x;
      accu [1] += cloud[indices[i]].x * cloud[indices[i]].y;
      accu [2] += cloud[indices[i]].x * cloud[indices[i]].z;
      accu [3] += cloud[indices[i]].y * cloud[indices[i]].y;
      accu [4] += cloud[indices[i]].y * cloud[indices[i]].z;
      accu [5] += cloud[indices[i]].z * cloud[indices[i]].z;
      accu [6] += cloud[indices[i]].x;
      accu [7] += cloud[indices[i]].y;
      accu [8] += cloud[indices[i]].z;
    }
  }

  accu /= static_cast<Scalar> (point_count);
 \end{lstlisting}
 
Por último, se almacenan en \textit{centroid} y \textit{covariance\_matrix} los resultados finales del algoritmo.
 
 \begin{lstlisting}[language=C++,breaklines]
  centroid[0] = accu[6]; centroid[1] = accu[7]; centroid[2] = accu[8];
  centroid[3] = 1;
  covariance_matrix.coeffRef (0) = accu [0] - accu [6] * accu [6];
  covariance_matrix.coeffRef (1) = accu [1] - accu [6] * accu [7];
  covariance_matrix.coeffRef (2) = accu [2] - accu [6] * accu [8];
  covariance_matrix.coeffRef (4) = accu [3] - accu [7] * accu [7];
  covariance_matrix.coeffRef (5) = accu [4] - accu [7] * accu [8];
  covariance_matrix.coeffRef (8) = accu [5] - accu [8] * accu [8];
  covariance_matrix.coeffRef (3) = covariance_matrix.coeff (1);
  covariance_matrix.coeffRef (6) = covariance_matrix.coeff (2);
  covariance_matrix.coeffRef (7) = covariance_matrix.coeff (5);

  return (static_cast<unsigned int> (point_count));
}
\end{lstlisting}

Se ha mencionado previamente que el corazón del algoritmo la estimación de vectores normales a la superficie lo conforman el cálculo de matrices de covarianzas y las componentes de cada vector normal.

Se ha decidido seleccionar el cálculo de la matriz de covarianzas explicado en la figura \ref{fig:compute_computeMean} para su optimización puesto que se reduce a operaciones con vectores y matrices que no requieren el uso de librerías específicas a diferencia del cálculo de las componentes del vector normal que necesita utilizar Eigen y complica la optimización.

\section{Conclusiones}



\chapter{Finalización del trabajo}
\section{Conclusiones y trabajo futuro}
En el presente trabajo se ha desarrollado en primer lugar un trabajo de investigación sobre el concepto de nube de puntos, los sensores que permiten crear nubes, sus aplicaciones y cómo trabajar con ellas mediante la librería PCL. Se ha profundizado en el proceso de alineamiento de nubes de puntos tratando la estimación de puntos clave de tipo SIFT y las normales a la superficie que la nube representa, todo ello mediante software, tanto el disponible en la librería PCL tanto el creado por el propio autor. Se han hecho las mediciones adecuadas para determinar que la estimación de vectores normales es la parte más costosa del proceso. A partir de dicha estimación, se ha seleccionado uno de sus algoritmos, puramente aritmético, que consiste en calcular matrices de covarianzas y centroides dados un conjunto de puntos distribuidos en un espacio tridimensional. A continuación, mediante la síntesis de alto nivel, se ha generado una IP de hardware digital que lleva a cabo el anterior algoritmo de forma acelerada. Además, se ha comprobado el correcto funcionamiento de la IP sobre un sistema embebido y se ha demostrado una reducción de más del 160\% de tiempo de ejecución del algoritmo ejecutado mediante hardware respecto a si se ejecuta en software.
\\
\\
El siguiente paso para tener un sistema de visión funcionando sobre el sistema embebido es integrar la IP sobre el mismo y hacer que el programa de cálculo de keypoints SIFT la utilice para estimar matrices de covarianzas y centroides en lugar de efectuar dicha operación mediante software.
\\
\\
Para ello, no hay que modificar el programa desarrollado por el autor del trabajo y ya explicado en el apartado \ref{extraccion_sift} sino que hay que modificar la propia librería de PCL, en concreto sustituir la llamada al método $ComputeMeanAndCovarianceMatrix$ explicado en el apartado \ref{normales_bajo_nivel} por la llamada a la ejecución de la IP hardware. Además, hay que crear las entradas adecuadas a esta IP, es decir, tres vectores de coordenadas X,Y,Z de cada punto de la nube así como un vector que contiene los índices que forman la vecindad en la iteración actual, tal y como se ha explicado en el apartado \ref{explicacion_software}
\\
\\
Con la integración de la IP en el sistema embebido se puede acelerar el proceso de cálculo de vectores normales a la superficie que representa una nube de puntos y que contribuye a la estimación de puntos SIFT en la misma y en última instancia al proceso de alineamiento de nubes. Así pues, un conjunto de nubes alineadas en una sola quedan preparadas para usarse como mapas del entorno próximo de vehículos autónomos, digitalización del terreno o creación de representaciones de objetos en forma de nubes de puntos con diferentes niveles de detalle y la posibilidad de reconstruir superficies.

\section{Organización del trabajo}
La duración total del proyecto abarca 11 meses: comienza en marzo de 2018 y finaliza en enero de 2019. Se estima que el autor del proyecto ha invertido de media 2.5 horas diarias lo que significa un total de 

$$11 meses * 30 \frac{dias}{mes} * 2.5 \frac{horas}{dia} = 825 horas$$

Sin embargo, se agrupan en dos meses, julio y agosto, los periodos de vacaciones, realización de exámenes y tiempo invertido en actividad profesional (prácticas extracurriculares durante el verano) así como cualquier otro periodo de inactividad. Esto significa que el número total de horas invertidas queda en:

$$9 meses * 30 \frac{dias}{mes} * 2.5 \frac{horas}{dia} = 675 horas$$

El conjunto de tareas realizadas son las siguientes, teniendo cada una su identificador numérico para el diagrama Gantt presentado:
\begin{itemize}
\item[1)] Estudio del concepto de nube de puntos, librería PCL y desarrollo del marco teórico
\item[2)] Desarrollo de objetivos y selección de herramientas
\item[3)] Desarrollo de pipeline de visualización y estimación de puntos SIFT
\item[4)] Medición de tiempos de ejecución de algoritmos de PCL
\item[5)] Análisis del algoritmo de extracción de normales
\item[6)] Síntesis de alto nivel para generación de IP hardware
\item[7)] Implementación y validación de la IP hardware
\item[8)] Desarrollo de la memoria
\end{itemize}

En el diagrama Gantt presentado se resaltan en verde las tareas relacionadas con el trabajo desarrollado por el autor y en rojo las tareas de investigación.


\begin{landscape}
\newpage

\begin{ganttchart}{1}{44}\label{gantt}

%titulo diagrama
\gantttitle{Diagrama Gantt para el desarrollo del trabajo}{44} \\[grid]

%meses
\gantttitle{Marzo}{4}
\gantttitle{Abril}{4}
\gantttitle{Mayo}{4}
\gantttitle{Junio}{4}
\gantttitle{Julio}{4}
\gantttitle{Agosto}{4}
\gantttitle{Septiembre}{4}
\gantttitle{Octubre}{4}
\gantttitle{Noviembre}{4}
\gantttitle{Diciembre}{4}
\gantttitle{Enero}{4}\\

%semanas
\gantttitle[
title label node/.append style={left=7pt and -3pt}
]{Semana: }{0}
\gantttitlelist{1,...,44}{1} \\


%tareas
\ganttbar[bar/.append style={fill=red!80}]{Tarea 1)}{1}{16} \\
\ganttbar[bar/.append style={fill=green!80}]{Tarea 2)}{7}{10} \\
\ganttlinkedbar[bar/.append style={fill=green!80}]{Tarea 3)}{9}{16}\\
\ganttlinkedbar[bar/.append style={fill=green!80}]{Tarea 4)}{13}{16}\\
\ganttlinkedbar[bar/.append style={fill=red!80}]{Tarea 5)}{25}{32}\\
\ganttlinkedbar[bar/.append style={fill=green!80}]{Tarea 6)}{32}{36}\\
\ganttlinkedbar[bar/.append style={fill=green!80}]{Tarea 7)}{36}{42}\\
\ganttbar[bar/.append style={fill=green!80}]{Tarea 8)}{13}{16}\\
\ganttlinkedbar[bar/.append style={fill=green!80}]{Tarea 8)}{25}{44}


\end{ganttchart}
\end{landscape}
\newpage
\section{Presupuesto}
Para la realización del presente trabajo se han utilizado tanto recursos materiales o hardware como software y propiedades intelectuales en forma de licencias además de las horas de trabajo y invertidas por el autor y el tutor. 
\\
\\
Se muestra en la tabla \ref{coste_material} el análisis económico de los recursos materiales lo que resulta en un total de $813 euros$ 
\\
\\
En el caso del software y licencias se tiene un coste nulo puesto que se ha utilizado software gratuito o bien se dispone de una licencia de estudiante. Se muestra el presupuesto de software y licencias en la tabla \ref{coste_software}
\\
\\
Por último, según el BOE del ministerio de empleo y seguridad social del año 2018\cite{BOE} se tiene en la tabla \ref{coste_humano} los salarios correspondientes a un ingeniero técnico e ingeniero. Se considera también una joranda laboral de 40 horas semanales o lo que es lo mismo, 8 horas diarias.
\\
\\
Con estos datos y considerando un número total de horas invertidas por el alumno y el tutor de 675 y 20, respectivamente se tiene un total de:

$$675 horas * 58.05 \frac{ euros }{dia} * \frac{1 dia}{8 horas} + 20 horas * 63.24 \frac{ euros }{dia} * \frac{1 dia}{8 horas} =  4897.96 euros + 158.1 euros = 5056.06 euros$$
\\
\\
\textbf{Por lo tanto, sumando los costes materiales, de software y humanos se estima una valoración del presente trabajo en $813 euros+ 5056.06 euros= 5869.06 euros$}
\begin{table}[!htb]
\begin{tabular}{cccc|c|}
\cline{2-5}
\multicolumn{1}{c|}{} & \multicolumn{1}{c|}{Recursos materiales}       & \multicolumn{1}{c|}{Coste unitario euros} & Cantidad              & Coste total euros \\ \cline{2-5} 
\multicolumn{1}{c|}{} & \multicolumn{1}{c|}{Pynq-Z1}                   & \multicolumn{1}{c|}{199}              & 1                     & 199           \\ \cline{2-5} 
\multicolumn{1}{c|}{} & \multicolumn{1}{c|}{Tarjeta micro SD}          & \multicolumn{1}{c|}{14}               & 1                     & 14            \\ \cline{2-5} 
\multicolumn{1}{c|}{} & \multicolumn{1}{c|}{Ordenador laboratorio CEI} & \multicolumn{1}{c|}{600}              & 1                     & 600           \\ \cline{2-5} 
\multicolumn{1}{l}{}  & \multicolumn{1}{l}{}                           & \multicolumn{1}{l}{}                  & \multicolumn{1}{l|}{} & 813           \\ \cline{5-5} 
\end{tabular}
\caption{Presupuesto de recursos materiales.}
\label{coste_material}
\end{table}

\begin{table}[!htb]
\begin{tabular}{cccc|c|}
\cline{2-5}
\multicolumn{1}{c|}{} & \multicolumn{1}{c|}{Recursos materiales}              & \multicolumn{1}{c|}{Coste unitario euros}       & Cantidad              & Coste total euros \\ \cline{2-5} 
\multicolumn{1}{c|}{} & \multicolumn{1}{c|}{PCL}                              & \multicolumn{1}{c|}{Gratuito}               & 1                     & 0             \\ \cline{2-5} 
\multicolumn{1}{c|}{} & \multicolumn{1}{c|}{Máquina virtual Ubuntu}           & \multicolumn{1}{c|}{Gratuito}               & 1                     & 0             \\ \cline{2-5} 
\multicolumn{1}{c|}{} & \multicolumn{1}{c|}{Vivado Design Suite HLx Editions} & \multicolumn{1}{c|}{Licencia de estudiante} & 1                     & 0             \\ \cline{2-5} 
\multicolumn{1}{l}{}  & \multicolumn{1}{l}{}                                  & \multicolumn{1}{l}{}                        & \multicolumn{1}{l|}{} & 0             \\ \cline{5-5} 
\end{tabular}
\caption{Presupuesto de recursos software y licencias.}
\label{coste_software}
\end{table}

\begin{table}[!htb]
\begin{tabular}{cccccll}
\cline{2-5}
\multicolumn{1}{c|}{} & \multicolumn{1}{c|}{Costes humanos}    & \multicolumn{1}{c|}{\begin{tabular}[c]{@{}c@{}}Salario  anual 2018 euros \\ 1,30 \%\end{tabular}} & \multicolumn{1}{c|}{Salario mes euros} & \multicolumn{1}{c|}{Salario día euros} &  &  \\ \cline{2-5}
\multicolumn{1}{c|}{} & \multicolumn{1}{c|}{Ingeniero técnico} & \multicolumn{1}{c|}{24.669,15}                                                                & \multicolumn{1}{c|}{1.762,08}    & \multicolumn{1}{c|}{58,05}       &  &  \\ \cline{2-5}
\multicolumn{1}{c|}{} & \multicolumn{1}{c|}{Ingeniero}         & \multicolumn{1}{c|}{26.876,99}                                                                 & \multicolumn{1}{c|}{1.919,78}    & \multicolumn{1}{c|}{63,24}       &  &  \\ \cline{2-5}
                      &                                        &                                                                                               &                                  &                                  &  &  \\
\multicolumn{1}{l}{}  & \multicolumn{1}{l}{}                   & \multicolumn{1}{l}{}                                                                          & \multicolumn{1}{l}{}             &                                  &  & 
\end{tabular}
\caption{Salarios de ingeniero e ingeniero técnico a 2018.}
\label{coste_humano}
\end{table}

\begin{thebibliography}{99}
%\bibitem {simple_vista} Clark, R.N., 2005. Notes on the Resolution and Other Details of the Human Eye. Clarkvision Photography - Resolution of the Human Eye. [ONLINE] Disponible en: \url{http://www.clarkvision.com/imagedetail/eye-resolution.html}. [Último acceso 22 julio 2018].

%\bibitem{angle_of_view} Wikipedia. 2003. Angle of view. [ONLINE] Disponible en: \url{https://en.wikipedia.org/wiki/Angle_of_view#Common_lens_angles_of_view}. [Último acceso 22 julio 2018]

%\bibitem {mit_experiment} Massachussetts Institute of Technology. 2014. In the blink of an eye. [ONLINE] Disponible en: \url{http://news.mit.edu/2014/in-the-blink-of-an-eye-0116}. [Último acceso 22 julio 2018].

%\bibitem {tiempo_reaccion} Fuerza y control. 2006. La velocidad de reacción: El tiempo de reacción simple, complejo y la anticipación. [ONLINE] Disponible en: \url{https://www.fuerzaycontrol.com/la-velocidad-de-reaccion-el-tiempo-de-reaccion-simple-complejo-la-anticipacion/}. [Último acceso 22 julio 2018].

%\bibitem {historia_tecnologia} Tecnomagazine. 2018. Historia de la Tecnología. [ONLINE] Disponible en: \url{https://tecnomagazine.net/2018/04/30/historia-de-la-tecnologia/}. [Último acceso 22 julio 2018].

%\bibitem {vision_artificial} Klette, R., 2014. Concise Computer Vision an Introduction into Theory and Algorithms, London: Springer London.

%\bibitem {ojo_humano} lhg-admin, 2012. How the Human Eye Works | Cornea Layers/Role | Light Rays. NKCF.org. [ONLINE] Disponible en: \url{https://www.nkcf.org/about-keratoconus/how-the-human-eye-works/}. [Último acceso 6 octubre 2018].


\bibitem {point_cloud} Gregorz Ciepka, 2018. What is a point cloud. 3d Laser Scanning. Disponible en: \url{https://www.3deling.com/whta-is-a-point-cloud/}. [Último acceso 6 octubre 2018].


\bibitem {point_cloud_rgb} Ewa, 2018. RGB and Intensity - Point cloud display options. 3d Laser Scanning. Disponible en: \url{https://www.3deling.com/rgb-point-cloud/}. [Último acceso 6 octubre 2018].

 

\bibitem {pcl_conejo_stanford} Greg Turk, The Stanford 3D Scanning Repository. Matt's Webcorner - Cloth. Disponible en: \url{http://graphics.stanford.edu/data/3Dscanrep/}. [Último acceso 6 octubre 2018].


\bibitem {pcd_exteriores} Nüchter, A., Robotic 3D Scan Repository. Disponible en: \url{http://kos.informatik.uni-osnabrueck.de/3Dscans/}. [Último acceso 6 octubre 2018].


\bibitem {escaner_riegl} MicroGeo, Escàner Làser 3D: Escàner Làser Riegl VZ-400. Laser Scanner 3D, Termografia Infrarosso, Topografia, Fotogrammetria | MICROGEO. Disponible en: \url{https://www.microgeo.it/es/escàner-làser-3d/modelos/scanner-riegl-vz400.aspx}. [Último acceso 6 octubre 2018].


\bibitem {escaner_cyberware} Cyberware, Model Shop Color 3D Scanner. Model Shop Color 3D Scanner (Model MS). Disponible en: \url{http://cyberware.com/products/scanners/ms.html}. [Último acceso 6 octubre 2018].


\bibitem {lidar} lidar uk, lidar-uk.com. [ONLINE] Disponible en: \url{http://www.lidar-uk.com/how-lidar-works/}. [Último acceso 6 octubre 2018].

\bibitem {control_laser} Ionix, Laser Marking. Ionix Oy. [ONLINE] Disponible en: \url{http://www.ionix.fi/en/technologies/laser-processing/laser-marking/}. [Último acceso 6 octubre 2018].


\bibitem {fotogrametria} James, David and Eckermann, Juergen and Belblidia, Fawzi and Sienz, 2015, "Point cloud data from Photogrammetry techniques to generate 3D Geometry"

\bibitem {canon} Canon España, Canon EOS 550D - Cámaras EOS: Réflex Digitales y Compactas de Sistema. Canon. [ONLINE] Disponible en: \url{https://www.canon.es/for_home/product_finder/cameras/digital_slr/eos_550d/}. [Último acceso 20 enero 2019].

\bibitem {agisoft} Agisoft Metashape. Orthophoto \& DEM generation. [ONLINE] Disponible en: \url{https://www.agisoft.com/}. [Último acceso 20 enero 2019].

\bibitem {vehiculos_autonomos} Vivien Potó, József Árpád Somogyi, Tamás Lovas y Árpád Barsi, "Laser scanned point clouds to support autonomous vehicles" [ONLINE] Disponible en: \url{http://www.sciencedirect.com/science/article/pii/S235214651731027X}. [Último acceso 20 enero 2019].


\bibitem {faro} User Manual for the Focus3D 20/120 and S 20/120. FARO® Knowledge Base. [ONLINE] Disponible en: \url{https://knowledge.faro.com/Hardware/3D_Scanners/Focus/User_Manual_for_the_Focus3D_20-120_and_S_20-120 }. [Último acceso 20 enero 2019].


\bibitem {riegl_450} Melguizo, J.P., RiEGL VMX 450. Laserscan Spain - RiEGL VMX-450 escáner lidar móvil. [ONLINE] Disponible en: \url{http://www.laserscan.es/productos/riegl-vmx-450/}. [Último acceso 20 enero 2019].



%\bibitem {kinect} Microsoft, Kinect for Windows. Home - Microsoft Graph. [ONLINE] Disponible en: \url{https://developer.microsoft.com/en-us/windows/kinect}. [Último acceso 6 octubre 2018].


\bibitem {realitycapture} 10x Faster Than Anything On The Market. Home - CapturingReality.com. Disponible en: \url{https://www.capturingreality.com/}. [Último acceso 6 octubre 2018].


\bibitem {PCL} PCL, PCL - Point Cloud Library (PCL). About - Point Cloud Library (PCL). Disponible en: \url{http://pointclouds.org/}. [Último acceso 6 octubre 2018].


\bibitem {PCL_modulos} PCL, Documentation. About - Point Cloud Library (PCL). Disponible en: \url{http://pointclouds.org/documentation/}. [Último acceso 6 octubre 2018].



\bibitem {eigen} Eigen. Disponible en: \url{ http://eigen.tuxfamily.org/index.php?title=Main_Page }. [Último acceso 6 octubre 2018].

\bibitem {flann} FLANN - Fast Library for Approximate Nearest Neighbors : FLANN - FLANN browse. Bayes' Rule. Disponible en: \url{https://www.cs.ubc.ca/research/flann/}. [Último acceso 6 octubre 2018].

\bibitem {boost} Boost C Libraries. Boost C Libraries. Disponible en: \url{ https://www.boost.org/}. [Último acceso 6 octubre 2018].


\bibitem {vtk} The Visualization Toolkit. VTK. Disponible en: \url{https://www.vtk.org/}. [Último acceso 6 octubre 2018].

\bibitem {PCD_extras} Point Cloud Library (PCL): pcl::PointCloud< T > Class Template Reference. Disponible en: \url{http://docs.pointclouds.org/trunk/classpcl_1_1_point_cloud.html}. [Último acceso 6 octubre 2018].


\bibitem {registration} Documentation. About - Point Cloud Library (PCL). Disponible en: \url{http://pointclouds.org/documentation/tutorials/registration_api.php}. [Último acceso 6 octubre 2018].

%%%%%%%%%%%%%%%%%%%%%%%CAPITULO 2%%%%%%%%%%%%%%%%%%%%%%%%%%%%

\bibitem {pynq}Xilinx, FPGA Design Flow Overview. Disponible en: \url{https://www.xilinx.com/products/boards-and-kits/1-hydd4z.html}. [Último acceso 6 octubre 2018].



\bibitem {puntos_clave} Documentation. About - Point Cloud Library (PCL). Disponible en: \url{http://pointclouds.org/documentation/tutorials/registration_api.php#registration-api}. [Último acceso 29 diciembre 2018].



\bibitem {puntos_clave_pwp} Federico Tombari, Keypoints and Features. Disponible en: \url{http://www.pointclouds.org/assets/uploads/cglibs13_features.pdf}. [Último acceso 29 diciembre 2018].



\bibitem {sift_opencv} OpenCV,  Introduction to SIFT (Scale-Invariant Feature Transform) Cascade Classification - OpenCV 2.4.13.7 documentation. Disponible en: \url{https://docs.opencv.org/3.0-beta/doc/py_tutorials/py_feature2d/py_sift_intro/py_sift_intro.html}. [Último acceso 29 diciembre 2018].


\bibitem {paper_registration} D. Holz, A. E. Ichim, F. Tombari, R. B. Rusu and S. Behnke, ``Registration with the Point Cloud Library: A Modular Framework for Aligning in 3-D'' in IEEE Robotics \& Automation Magazine, vol. 22, no. 4, pp. 110-124, Dec. 2015. Disponible en: \url{http://ieeexplore.ieee.org/stamp/stamp.jsp?tp=&arnumber=7271006&isnumber=7349124
}. [Último acceso 29 diciembre 2018].


\bibitem {ubuntu} Canonical, Six reasons why developers choose Ubuntu Desktop. ubuntu. Disponible en: \url{https://www.ubuntu.com/}. [Último acceso 30 diciembre 2018].

\bibitem {virtualbox} Oracle VM VirtualBox. Disponible en: \url{https://www.virtualbox.org/}. [Último acceso 30 diciembre 2018].

\bibitem {pcl_installation} Prebuilt binaries for Linux. PCL - Point Cloud Library (PCL). Disponible en: \url{http://www.pointclouds.org/downloads/linux.html }. [Último acceso 30 diciembre 2018].

\bibitem {vivado_descarga} Downloads. FPGA Design Flow Overview. Disponible en: \url{https://www.xilinx.com/support/download.html}. [Último acceso 30 diciembre 2018].




%%%%%%%%%%%%%%%%%%%%%%%CAPITULO 3%%%%%%%%%%%%%%%%%%%%%%%%%%%%

\bibitem {ejecutable} Documentation. About - Point Cloud Library (PCL). Disponible en: \url{http://pointclouds.org/documentation/tutorials/using_pcl_pcl_config.php#using-pcl-pcl-config}. [Último acceso 30 diciembre 2018].


\bibitem {tutoriales} Documentation. About - Point Cloud Library (PCL). Disponible en: \url{http://pointclouds.org/documentation/tutorials/}. [Último acceso 30 diciembre 2018].

\bibitem {modulo_io} Documentation. About - Point Cloud Library (PCL). Disponible en: \url{http://pointclouds.org/documentation/tutorials/reading_pcd.php#reading-pcd}. [Último acceso 30 diciembre 2018].

\bibitem {narf} Documentation. About - Point Cloud Library (PCL). Disponible en: \url{http://pointclouds.org/documentation/tutorials/narf_keypoint_extraction.php#narf-keypoint-extraction}. [Último acceso 30 diciembre 2018].

\bibitem {ejemplo_visualizacion} Documentation. About - Point Cloud Library (PCL). Disponible en: \url{http://pointclouds.org/documentation/tutorials/cloud_viewer.php#cloud-viewer}. [Último acceso 16 enero 2019].

\bibitem {ejemplo_narf} Documentation. About - Point Cloud Library (PCL). Disponible en: \url{http://pointclouds.org/documentation/tutorials/narf_feature_extraction.php#narf-feature-extraction}. [Último acceso 16 enero 2019].

\bibitem {sift_class} Point Cloud Library (PCL): pcl::PointCloud< T > Class Template Reference. Disponible en: \url{http://docs.pointclouds.org/1.8.1/classpcl_1_1_s_i_f_t_keypoint.html}. [Último acceso 16 enero 2019].


\bibitem {normal} Documentation. About - Point Cloud Library (PCL). Disponible en: \url{http://pointclouds.org/documentation/tutorials/normal_estimation.php}. [Último acceso 30 diciembre 2018].


\bibitem {extraccion_normales} Documentation. About - Point Cloud Library (PCL). Disponible en: \url{
http://pointclouds.org/documentation/tutorials/normal_estimation.php#normal-estimation}. [Último acceso 30 diciembre 2018].


\bibitem {kdtree} Documentation. About - Point Cloud Library (PCL). Disponible en: \url{http://pointclouds.org/documentation/tutorials/kdtree_search.php#kdtree-search}. [Último acceso 30 diciembre 2018].


\bibitem {sift_escala} Sinha, U., The scale space. AI Shack. Disponible en: \url{http://aishack.in/tutorials/sift-scale-invariant-feature-transform-scale-space/}. [Último acceso 30 diciembre 2018].


\bibitem {gaussiana}Dr. Edmund Weitz, SIFT - Scale-Invariant Feature Transform. Disponible en: \url{http://weitz.de/sift/}. [Último acceso 30 diciembre 2018].




%%%%%%%%%%%%%%%%%%%%%%%CAPITULO 4%%%%%%%%%%%%%%%%%%%%%%%%%%%%

\bibitem {ctime}(time.h). cplusplus.com. Disponible en: \url{http://www.cplusplus.com/reference/ctime/}. [Último acceso 31 diciembre 2018].


%%%%%%%%%%%%%%%%%%%%%%%CAPITULO 5%%%%%%%%%%%%%%%%%%%%%%%%%%%%

\bibitem {normales_extra}Rusu, Radu Bogdan - "Semantic 3D Object Maps for Everyday Manipulation in Human Living Environments", noviembre 2010, páginas: 45 - 50 Disponible en: \url{http://mediatum.ub.tum.de/doc/800632/941254.pdf}. [Último acceso 31 diciembre 2018].


\bibitem {pclbase} Point Cloud Library (PCL): pcl::PointCloud< T > Class Template Reference. Disponible en: \url{http://docs.pointclouds.org/1.8.1/classpcl_1_1_p_c_l_base.html}. [Último acceso 31 diciembre 2018].

\bibitem {normal_estimation} Point Cloud Library (PCL): pcl::PointCloud< T > Class Template Reference. Disponible en: \url{http://docs.pointclouds.org/1.8.1/classpcl_1_1_normal_estimation.html}. [Último acceso 31 diciembre 2018].

\bibitem {feature} Point Cloud Library (PCL): pcl::PointCloud< T > Class Template Reference. Disponible en: \url{http://docs.pointclouds.org/1.8.1/classpcl_1_1_feature.html}. [Último acceso 31 diciembre 2018].


\bibitem {calculo_covarianzas} Point Cloud Library (PCL): pcl::PointCloud< T > Class Template Reference. Disponible en: \url{http://docs.pointclouds.org/1.8.1/centroid_8hpp_source.html#l00489}. [Último acceso 31 diciembre 2018].


\bibitem {calculo_compute_point_normal} Point Cloud Library (PCL): pcl::PointCloud< T > Class Template Reference. Disponible en: \url{http://docs.pointclouds.org/1.8.1/classpcl_1_1_normal_estimation.html#abd83b3241b46704f1ffc424dee2aa031}. [Último acceso 31 diciembre 2018].


%%%%%%%%%%%%%%%%%%%%%%%CAPITULO 6%%%%%%%%%%%%%%%%%%%%%%%%%%%%





\end{thebibliography}



\end{document}
