\chapter{Extracción de normales a la superficie de una nube usando PCL}
\section{Introducción}
Se ha justificado en el capítulo anterior que será el algoritmo de extracción de vectores normales el que se llevará a hardware digital para ser optimizado. 

En este capítulo, se va a estudiar en profundidad dicho algoritmo exclusivamente en el ámbito de la librería PCL ya que ésta se sirve de librerías externas como son principalmente Eigen y boost. Este estudio es necesario para comprender cómo funciona el algoritmo y así poder realizar la optimización del hardware digital obtenido a partir de éste.

Por lo tanto, se explicará tanto en alto como en bajo nivel cómo PCL estima normales a la superficie de una nube de puntos de un modo semejante al que se ha utilizado para explicar fragmentos de código en capítulos anteriores.



\section{Estimación de normales a la superficie de una nube: alto nivel}

\section{Estimación de normales a la superficie de una nube: bajo nivel}


\section{Conclusiones}
...

En el siguiente capítulo, se mostrarán las modificaciones pertinentes al algoritmo de estimación de normales para que sea sintetizable en hardware y se explicará el proceso de optimización llevado a cabo.